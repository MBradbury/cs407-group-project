% !TeX root = Report.tex
\section{Introduction}

\subsection{What is a Wireless Sensor Network?}

A wireless sensor network, or WSN for short, is a collection of networked sensors called Motes; these sensors are capable of short range wireless communication and they have the ability to sense their surrounding environment\cite{Mica2002,TankBible}. The network forms a distributed system that can perform a variety of distributed algorithms, usually data gathering and similar tasks. To communicate, each node is equipped with a radio that allows them to send and receive messages to neighbouring nodes within a limited range. To sense the environment motes typically have a range of embedded sensors such as heat, light, humidity and many others. They typically contain a simple central processing unit (CPU), which is programmed to control the hardware on the motes. The CPU also processes events which are triggered by the hardware (such as messages being sent and received) and it also handles any other computation necessary for the operation of the system. As the platform is designed to be mobile the motes do not operate on a mains power supply, instead they run off stored energy in a battery. WSNs is a field of Computer Science that is currently the focus of much research and wireless sensor networks have a wide range of practical applications that stretch from battlefield intelligence for the military\cite{Akyildiz2002393,1368897,1457970} to industrial process monitoring for manufacturing companies\cite{?}.

A defining characteristic of designing applications for wireless sensor networks is the restricted and finite energy supply available to each node. Therefore, wireless sensor nodes tend not to use expensive broadcasting protocols such as IEEE 802.11 \cite{Mica2002}, but instead use much simpler alternatives to save energy. For example wireless protocols such as IEEE 802.15.4 ZigBee \cite{1253873, 4014617} are designed to be used by wireless sensor networks and have a lower energy usage associated with them. Some applications rely on even lower level behaviour specified by a certain MAC layer \cite{5751321,4469515,Polastre:2004:VLP:1031495.1031508,1019408,Buettner:2006:XSP:1182807.1182838}, these applications involve a trade-off between development time and energy usage. Where simplicity is often sacrificed for decreased energy usage. Using these simple protocols unfortunately has the downside of meaning that broadcasts are subject to several types of collisions and message losses. So it is very important that the software running on the nodes is designed to handle these cases. Being battery powered means that development of applications for Wireless Sensor Networks is fundamentally limited to maximising the system's lifetime so that the highest utility can be achieved from the network.

As wireless sensor nodes operate in harsh outdoors condition\cite{SzewczykPMC04, Werner-Allen:2006:FYV:1298455.1298491}, there is a high probability of them failing. These faults can range from hardware damage caused by environmental conditions or tampering, software bugs, or simply a denial of service caused by nodes running out of power. So algorithms and software need to be designed to handle these potential failures, otherwise they risk catastrophic failure when they encounter these issues.

Wireless sensor nodes are designed with the intention for them to operate in remote and traditionally unreachable locations with no human input for the lifetime of their operation\cite{1437066}. Given this a defining characteristic of a WSN system is that any applications developed for it must be self-configuring in nature; this is, the system must be able to organize itself and the network with no external input\cite{1368897}. Evidently, solutions to this issue are often considered hand in hand with the problem of network robustness and fault tolerance.   

While a limited energy source, self-configuration and network robustness are the predominant characteristic of a wireless sensor node, there are numerous other traits or issues that can be considered. For instance it is possible for these nodes to be mobile (for example an ad-hoc network of PDAs or motes built into soldiers helmets) \cite{4224091}, this can lead to very interesting behaviour in handling communication between these nodes.

%TODO: MORE OBSCURE WSN BEHAVIOUR CONSIDERATIONS

\subsection{The Problem - Debugging Distributed Systems}

Developing a distributed system is considered a particularly challenging task, more so than a traditional application, there are several reasons for this. Firstly, Within a distributed system multiple processes must execute in parallel, this means that variables may be updated independently or in repose to other processes which can lead to a myriad of synchronisation and timing issues that the developer must account for. Secondly, traditional programming languages are not well suited to develop distributed programs\cite{93692,345131}.

In any system, software or otherwise, developed by humans there is the potential for mistakes. Mistakes can be benign or they can cause unintended behaviour and system failures. Developing tools to detect these \emph{bugs} and notify the developer so they can be corrected is an incredibly important part of any toolchain. For example the GNU toolchain has utilities such as gdb\cite{?}, this allows developers to place breakpoints in code which will halt the programs execution at that state so that it can be examined. There are also numerous other tools that look for memory issues (valgrind\cite{Bond:2007:TBA:1297105.1297057,Nethercote:2007:SBM:1254810.1254820,seward2004valgrind}), security flaws (TOOL NAME\cite{?}) and many other classes of bugs.

Developing distributed systems is a difficult task; however, debugging these systems can be even more challenging\cite{345131}. When considering a distributed system if you want to examine the state of a system at a given point you cannot simply set a breakpoint in your local binary. The solution to this debugging is non-trivial, this is due to the difficulties that arise from distributed systems being non-deterministic in nature due to message communications \cite{1676929,Joyce:1987:MDS:13677.22723,Fagerstrom:1988:DTD:55823.55833}. Be it when the message started transmission, how long it took, if it succeeded or in what order transmissions occurred. So every time a distributed program is run it is possible for a different result to be obtained, due to the different order of execution. This goes against one of the usual assumptions of debugging traditional applications where it is assumed that one execution with a set of inputs will execute in exactly the same way again with the same set of inputs \cite{?} (i.e. determinism).

As the execution may be different each time in a distributed system it is not suitable to wait for a bug to occur, and then try to work out where it is. Rather, the system needs to be self-evaluating its state as it executes the distributed program; if a fault is detected, then the debugging tools will report the issue. One way to do this it to test if the system satisfies some global predicate, of which there has been much work to find and check different classes of these predicates \cite{553309,345831,277788}. However, of all the work that has been done, little of it has focused on wireless sensor networks where an important focus is perhaps the trade-off between accuracy and the report-ability of a predicate with the aim of reducing energy usage. In this paper we will discuss our development of just such a set of tools, we intend to focus on developing a system that can accurately evaluate predicates and provide useful information about real sensor networks running outside of a simulator.

\subsection{Related Work}

\subsubsection{Fault-Error-Failure Cycle}
%TODO: Joe: Needs Work

It should be clear that the types of predicates are important to consider when developing a predicate checking mechanism. Also important are the types of errors that these predicate checking algorithms can detect. To understand this it is first important to understand how errors can arise, which can be done by examining the fault-error-failure cycle. This cycle says that a \emph{fault} once caused by either some external influence (e.g. radiation leading to bit-flips in memory \cite{1017791}) or internal influence (e.g. code bugs) will lead to an \emph{error}, this is the \emph{activation} step. An error is the manifestation of the fault (e.g. memory holding the incorrect value). An error then leads to a \emph{failure} in the step called \emph{propagation}, the failure of the system is an observable deviation from the system's specification (e.g. allowing doors to be opened that should remain closed). It is not always the case the faults lead to errors, or errors lead to failures, sometimes multiple faults or errors are respectively required to cause a single error or failure. \cite{1335465}

There is a choice of what should be measured and checked in predicates; should faults, errors or failures be measured? Faults cannot be measured \cite{?} in a way that is possible for a mote, so they are discounted. That leaves measuring errors and failures. Typically failures would be the event being measured \cite{?} as that is what arises after an error actually causes the issue to happen. However, errors can also be measured if there is a dedicated program checking the state and comparing it to an expected state \cite{?}. For example an ECC (error correcting code) such as a Hamming Code can be used to detect and correct an error (in this case a bit-flip) in some memory after a fault (such as a voltage surge ) \cite{hamming1950error}.

Much of what has been discussed has involved transient faults such as those caused by environmental conditions, however, there is a class of faults that are a lot more common and much easier to resolve - faults caused by software bugs. These faults can lead to programs ending up in the wrong state and performing incorrectly. There has been a certain amount of work that looks into detecting traditional distributed system bugs (such as deadlock \cite{5587352,5284172}) in wireless sensor networks. However, there has been little work in looking into providing tools to aid in system debugging.

\subsubsection{Classes of Distributed Predicates}
%TODO: Joe: Needs Work

To begin with it is important to understand what predicates are relevant to distributed systems. First off we have a distinction between global and local predicates, global predicates involve taking a consistent global snapshot of the system and checking whether the snapshot satisfies the global predicate \cite{277788} and local predicates instead work with a subset of the network \cite{553309}. These predicates have also have a notion of stability, a stable predicate will remain true once it has turned true (e.g. termination), whereas unstable predicates can alternate between true and false. Finally there is a distinction between weak and strong, where a weak predicate holds if there exists an observation where the predicate is true and a strong predicate holds if it is true for all observers of the distributed computation\cite{553309,Cooper:1991:CDG:127695.122774}. Knowing what classes of predicates there are is important because when checking certain properties of a system a certain class of predicate will be required and thus a certain implementation will be needed to ensure the predicate is correctly checked. An example of this is when running an algorithm using global snapshots to detect stable predicates, that same application may not be suitable to detect unstable predicates because the predicate could switch to false and then back to true before the next snapshot.

\subsubsection{Existing Sensor Network Predicate Checking Tools}

There are a number of existing predicate checking solutions that have already been developed that are more practical focused than the aforementioned theoretical work into global predicate detection.

\paragraph{H-SEND} One of these solutions is H-SEND \cite{herbert2007adaptive}, which stands for Hierachichal SEnsor Network Debugging. H-SEND is a framework for detecting faults in sensor networks, it was designed to minimise energy consumption and be capable of handling very large networks. As part of the implementation developer must specify invariants within their code using a custom grammar, these invariants are then semi--automatically inserted during compilation. If an invariant is violated at runtime actions are taken (such as increased logging frequency, or an error message to the base station), using these responses developer can use the information to fix the software, which possibly include uploading a patched version of the firmware.

Of the invariants that can be specified, there are typically three different dichotomies: (i) Local vs. Multi--node, (ii) Stateless vs. Stateful and (iii) Compile--time vs. Run--time. The first indicates whether the predicate needs information about the node it is being evaluated on (Local) or other nodes in the network (Multi). The second is if the invariant depends on the node's execution state (Stateful) or if it doesn't (Stateless). The third indicates if the invariant involves values that are fixed at compile time (such as integer constants) or if it compares against values obtained during run-time (such as neighbouring states or previous states).

H-SEND is optimised for WSNs in a variety of ways. For example, it minimises overhead by buffering messages it needs to send, and piggybacking them on the existing network traffic. Due to the hierarchical nature of the protocol, multinode-invariants can be checked efficiently at the closest parent node with all the required information.

\paragraph{Sympathy} One of the projects that is summarised by H-SEND's authors \citeauthor{herbert2007adaptive} in \cite{herbert2007adaptive} is a method for identifying and localizing failures called Sympathy \cite{ramanathan2005sympathy}. Sympathy is intended to be run either in pre- or post-deployment environments where it collects data from distributed nodes at a sink. When insufficient data is received it is taken to imply that there exists a problem (insufficient data is component defined). The idea is that by monitoring data (both actively and passively) between components the system can identify what kind of failure occurred in a certain area of the network. Both of which are very useful when trying to debug a failure.

It does, however, have some downsides. The first is that there is assumed to be no traffic and thus no application traffic or network congestion. These are real issues especially when applying this kind of debugging to a high throughput sensor network. There are also a number of spurious failure notification, which the authors are working on reducing, by applying a Bayes engine.

\paragraph{DAIKON} Following Sympathy's attempts to implement a Bayes engine to allow learning to better help classify response messages, there has been work on being able to automatically detect invariants in a system. DAIKON \cite{daikon} is a system that uses execution traces to produce a list of likely invariants by way of automatically inferring the invariants through the use of a prototype invariant detector.

The set of dynamically detected invariants depend on observed values and the invariants are an indication of the quality of a test suite. Daikon consists of two parts; a language specific front end and a language independent interference engine. The former executes the running program and accesses its runtime state to get the required information (consist of variables and their value). This information contains a subset of relevant variables and only these are written to the trace file. A variable is considered as relevant depending on the type of invariants targeted and whether they are accessible at an instrumentation point. This forms an input to the second part of the system - the invariant inference engine - which uses machine learning techniques and produces a set of detected invariants for the program.

The main challenge with these techniques is deducing the relevance of the invariants as it depends on the programmer's experience and knowledge of the underlying system. However, these can be improved with techniques like exploiting unused polymorphism and suppressing invariants that are logically implied by other invariants. DAIKON does not require the programmer to specify invariants for the application, however, it is not designed for distributed or resource-constrained systems like WSN.

\paragraph{DIDUCE} One of the issues with DAIKON is that it only detects possible invariants, DIDUCE \cite{diduce} (Dynamic Invariant Detection $\cup$ Checking Engine (DIDUCE)) uses a similar metholody to detect the invariants, but it also evaluates them as they are detected. Machine learning is employed to dynamically generate hypotheses of invariants for a system at run-time. The invariants begin extremely strict, and are relaxed over time to allow for new correct behaviour. The machine learning aspect means that developers do not have to specify invariants themselves (as compared to H-SEND), which proves beneficial as accurately pinpointing the values necessary for fault-free operation is non-trivial \cite{?}. DIDUCE checks against the invariants continually during a program's operation and reports all violations detected at the end of the run, whereas DAIKON merely presents the user with invariants found. For all its apparent usefulness, unfortunately DIDUCE was designed for large, complex systems rather than lightweight distributed systems with constrained resources such as sensor networks.

\paragraph{NodeMD} An alternative to debugging compared to either specifying a predicate or an invariant, or using machine learning to learn what to check for is to instead look for the faults that arise after a failure has occurred. This is the approach that NodeMD \cite{NodeMD} takes, that by looking for the faults that can cause undesired behaviour bugs in the system can be identified. NodeMD supports checking a number of fault classes: stack overflow, livelock, deadlock and application-specific faults. By having an extensible framework, developers of a system can write their own fault detectors and plug them into NodeMD's framework. The authors of NodeMD point out that ``human interaction is often the only reliable way to address many software issues'', therefore, NodeMD supports recording events that occur to humans can analyse them to find out how a failure occurred. To optimise this format for sensor network, the event are stored in a custom binary format to save space and reduce the number of messages to transmit it (if it is possible to transmit).

NodeMD also has support for a number of useful features to aid in debugging, the first is a debug mode that is entered when a fault is detected. The debug mode freezes critical parts of the system to prevent the fault from leading to errors. This prevents events such as a context switch after a stack overflow that would end up being performed incorrectly. The debug mode also resets certain OS components to a safe state, so that some components (such as the radio) are usable to report the fault that was detected.

The other two useful features are support for remote debugging and an implementation of dynamic reprogramming algorithms to update the firmware across the network. The remote debugging feature allows a human to access all the available fault information of a sensor node. Parameters can be changed to expose more information when the node is queried and the potentially useful feature of telling the node to restart is also available. Overall NodeMD provides lots of insight into the low level failures in wireless sensor networks.


\subsubsection{Practical Sensor Network Deployments}

In order to understand how these applications may be useful it is necessary to consider how sensor networks are actually used. Overall, wireless sensor networks occupy a niche market of monitoring and reporting on vast areas. The software is very careful to minimise energy consumption to maximise the lifetime of the network \cite{?} and it is also designed to operate on hardware with limited resources (such as memory) \cite{?}. The following are a few examples of real-world deployments and the experiences obtained in doing so.


\paragraph{Habitat Monitoring} Habitat monitoring is widely thought to be one of the key wireless sensor network application that is driving research and adoption. The problem involves determining the location of the sensor nodes, which then allows tracking \cite{Cerpa:2001:HMA:844193.844196} and two of the primary problems in sensor networks: data aggregation and energy efficient multihop communications.

The research undertaken by \citeauthor{SzewczykPMC04} in \cite{SzewczykPMC04} involves deploying sensor nodes on the Great Duck Island in order to test the long term real-world deployment of sensor networks. The importance of deploying and testing a WSN in the real world is due to the fact that challenges are encountered that are not present in indoor deployments or simulations. The problem becomes one of not just what software needs to run and how, but also consider on what hardware in what conditions. For example the authors wished to waterproof the mote in order to ensure that it survived dew, rain and flooding. However, when enclosing the motes it was worried that (i) radio transmissions would be interfered with and (ii) sensor data could be affected.

Overall, the deployed motes performed well (logging 1.1 million readings over 123 days), however, there were a number of abnormal results that included: (i) sensor readings outside their range, (ii) ``erratic packet delivery'' and (iii) failure of motes. The authors point out (in section 2.5) that in the future it would be useful to augment applications with the ability to notify that failures occurred and to perform self-healing.

The performance of the network showed that initially packet loss was up to 20\% but that slowly decreased, most likely due to the fact that the size of the network network was reducing over time (as nodes permanently crashed). Due to the low network utilisation (under 5\%) \citeauthor{SzewczykPMC04} initially believed that collisions would not play an important role, however, their results suggested otherwise. Their results suggest that this was caused by clock drift which lead to slot assignments not preventing collisions within their period. This importance of clock drift on TDMA-based MAC protocols is something that would tend not to be found from testing in a simulator because of the slow speed of simulation and the length of time it takes for these small clock drifts to lead to an effect.

Overall, one of the major conclusions of this work is that anomalous sensor data can be used to predict mote failures with a high degree of accuracy. The authors believe that this prediction allows for a high level of pro-active maintenance, and self-organisation and self-healing of the network.



% https://www.princeton.edu/~mrm/ZNetASPLOS.pdf
% http://citeseerx.ist.psu.edu/viewdoc/download?doi=10.1.1.14.6467&rep=rep1&type=pdf
\paragraph{Animal Monitoring} Instead of monitoring the habitat, why not simply directly monitor the animal in question? This is the track that a number of other wireless sensor networks have taken \cite{Juang:2002:ECW:635508.605408}. The traditional technique was to attach collars that emit radio waves that are used in manual triangulation or take GPS measurements and then recover the hardware attached to the animal to extract the data. Sensor networks provide a way to extract this data automatically in a much easier manner and can also allow for data to be accessed earlier.   Much of what is true for habitat monitoring is also true for monitoring animals: the hardware needs to be protected against the weather, the data needs to be extracted as reliably as possible and the battery should last as long as possible to make the deployment cost effective.

ZebraNet by \citeauthor{Juang:2002:ECW:635508.605408} in \cite{Juang:2002:ECW:635508.605408} investigates these issues as well as some of the unique deployment issues related to mounting hardware on animals. For example it is very important that the hardware is light enough for the animal to handle for a long time. This adds a new type of trade-off, when considering energy usage and reliability system developers must also consider weight.

Aside from the usual issues associated with sensor networks, having sensors attached to animals presents another very important problem that needs to be solved. Many of the animals that we wish to monitor, are desired to monitor for very good reasons. Most often it is the case that they are endangered or are often the victims of poaching \cite{6115161,Biagioni02theapplication}. By attaching wireless devices that broadcast information, it has been found that attackers can trace their way back to a source. There has been much research to develop ways to mitigate this problem, one example is using fake sources to lure an attacker in an alternate direction than that of the real source \cite{Ozturk:2004:SPE:1029102.1029117,Kamat.1437121,6296046}. These additional problems that arise as side-effects of communicating through a wireless medium are often difficult and expensive (in terms of energy) to overcome.



% http://static.usenix.org/events/mobisys06/full_papers/p28-hartung.pdf
\paragraph{Forest Fires} While habitat monitoring has been at the forefront of environmental monitoring, there is also a need to measure weather conditions in order to predict wildland fires that can occur across the globe \cite{libeliumForestFires,FireWxNet,4428702}. Fire behaviour can change drastically, depending on different environmental factors and topological features such as elevation and aspect. It is important for such behaviour to be predicted accurately and wildland firefighters are currently using means like observing current weather conditions and weather forecasts provided by the National Weather Service to do so. While data provided by the weather forecast gives a general overview of fire behaviours, those solely collected by the firefighters (e.g. using a belt-weather kit) only provides data for a sparse number of regions which is inadequate in making a full evaluation. With wireless sensor networks, firefighters can safely measure weather conditions over a wide range of locations and elevations anywhere within the fire environment.

FireWxNet by \citeauthor{FireWxNet} in \cite{FireWxNet} is a robust multi-tiered portable wireless system, designed to be deployed in a rugged environment. Compared to other application deployments, FireWxNet covers a unique topology which ranges from substantial and sharp elevation differences to an extremely wide coverage area of about 160 square kilometres. Wide area communication coverage and fine-grained local weather sensing coverage are some of the main challenges that was faced by the team. 

Some of the network challenges was ensuring a high-probability of receiving data from the sensors, regardless of interference and asynchronous links and this was tackled by using a best-effort converge-cast protocol. Other issues included verifying the connectivity between nodes and complications were a result of the sparse nature of the deployment and large changes in elevation between nodes. In addition, a user could not receive verification when adding a node to a currently deployed network and required the resetting of a neighbour node to gain connectivity.

When testing the performance of the network, results showed that the networks which were deployed on different mountains obtained an average yield of only 40\% and a 78\% unique yield. This may have been the effect of timing limitations, for example some sensors required some settling time once it was activated before it could produce any accurate readings. The CSMA MAC protocol which would back-off in the presence of interference may have also contributed to the number of packets sent. To improve the success rate, they chose the option of resending the packets multiple times, but at the cost of some efficiency. In the future, the team aims at designing protocols such as hop-to-hop ACKS, which hopes to cut down on the number of packet they send.



\paragraph{Volcano Monitoring} Active volcanoes need to be monitored to predict the likelihood of future eruptions so that early-warning signs can be issued for the evacuation of habitants near the volcano. Today, volcanoes use wired arrays of sensors such as seismometers and acoustic microphones to collect seismic and infrared (low-frequency acoustic) signals and determining a wide range of factors such as sources of volcanic eruptions, interior structure of a volcano and differentiating eruption signals from noise. A typical study can include the placement of several stations around the volcano where each contains a low distribution of wired sensors that collects data to a hard drive. Data is then collected manually in a possibly inconvenient location.

To address the issue of high power consumption of these wired sensors, \citeauthor{Werner-Allen:2006:FYV:1298455.1298491} in \cite{Werner-Allen:2006:FYV:1298455.1298491} introduced the deployment of embedded wireless sensor networks consisting of low-power nodes with minimum CPU, memory and wireless communication capabilities. This allows for long distance real time monitoring of volcanic activities and reducing the need for manual data collection. The main challenge however is that environmental monitoring studies sample at a frequency which is much lower than the sample rate of volcanic time series. This is due to the limited radio bandwidth of the sensors thus requiring the need for efficient power management techniques and accurate time synchronization of the nodes. 

\citeauthor{Werner-Allen:2006:FYV:1298455.1298491} implemented a wireless sensor network that was deployed on the Volcano Tungurahua in central Ecuador using a Mica2 sensor mote platform and three infrasonic microphone nodes. These nodes transmitted data to an aggregation node, which relayed the data over a 9km wireless link to the base station. Time synchronization for the infrasonic sensors was done using a GPS receiver which receives a GPS time signal and relays the data to the infrasound and aggregator nodes. Over a period of time, the temperature and battery voltage may change and this may affect the sampling rate of the individual nodes and the precision of the time recorded from the GPS time stamp message. These uncertainties was addressed by applying a linear regression to the logged data stream giving an estimation of the node outputs. In addition, the sensor nodes where required to be protected against harsh environments such as rain and long exposure of sunlight by using waterproof pelican cases and 1/4-wave whip antennas.

Their first deployment on the volcano included a small network where nodes could send continuous signals to each other, however this was not feasible for a larger network deployed over longer periods of time. To solve the issue with bandwidth and energy consumption, the team implemented a distributed event detector which only transmits well-correlated signals to the base station. A local event detection algorithm is used to trigger data collection for uncorrelated signals.

\begin{itemize}
\item Distributed Event Detector

To measure the correlated signal, the distributed detector uses a decentralized voting system among a group of nodes. Each nodes samples data at a continuous rate of 102.4Hz and buffers a window of such data while running a local event detection algorithm. If an event is triggered it uses a local radio broadcast to broadcast a vote to other nodes and a global flood to initiate global data collection from all the nodes. Radio contention is reduced using a token-bases scheme for scheduling transmission, where each nodes transmits their buffer one after the other. 

\item Local Event Detector

The team implemented two local event detectors; a threshold-based detector and an exponentially weighted moving average (EWMA)-based detector. The former is triggered when the signal rises above an upper threshold and below a lower threshold. However due to spurious signals such as wind noise, the detector may be susceptible to false trigger. On the other hand, the EWMA detector calculates two moving averages with different gain parameters and is triggered if the ratio of the two averages and the new sample exceeds some threshold T. This method is less affected by node sensitivity and any duplicate triggers over a window of 100 samples are suppressed.
\end{itemize}

\paragraph{Data Collection and Performance}

During the deployment, the team managed to log 54 hours of continuous data which includes seismoacoustic signals from several hundred events. The raw data was used to analyse the system's performance and many challenges where discovered in the early observations. They discovered that on average only 61 percent of data was retrieved from the network and on several occasions the modems which transmitted data back to the station would experience short drop-outs, causing all data aggregated from the nodes to be lost. In addition, duplicate packets were also recorded as a likely result of redundant retransmission and lost acknowledgement. Both lost and duplicate data had to be accounted for before being stored in the timebase.

\paragraph{Conclusion and Improvements}

In general, the event-triggered model was successful in detecting eruptions and other volcanic events and was able to verify the working of the local and global event detectors by examining the downloaded data. However, since the experiment was deployed over a limited area space of the volcano, further work is to be done to instrument volcanoes at a larger scale. This includes, and most importantly, the management of energy and bandwidth usage of the sensors and increasing its computational power to deviate from continuous data collection to enabling the collection of well-correlated signals. 



% http://ieeexplore.ieee.org/stamp/stamp.jsp?tp=&arnumber=5508230
% Drawbacks on conventional methods: http://www.eecs.berkeley.edu/~binetude/work/report.pdf
\paragraph{GENESI} Another class of application for wireless sensor networks is the structural health monitoring (SHM) of critical infrastructure such as bridges, tunnels and dams, to estimate the state of structural health or detecting the changes in structure that may affect its performance. The conventional method uses PCs wired to piezoelectrical accelerometers and has drawbacks such as (i) use of long wires over the structure, (ii) high cost of equipment, and (iii) expensive to install and maintain. Since SHM requires high data rate, large data size, and a relatively high duty cycle, it would be more efficient to use a WSN which would provide the same functionality as conventional methods but at a much lower price and permits denser monitoring.

\citeauthor{5508230} in \cite{5508230} developed GENESI (Green sEnsor NEtworks for Structural monItoring), which aims to efficiently harvest and generate energy from multiple-sources with radio triggering capabilities and using new algorithms and protocols to dynamically allocate the sensing and communication of tasks to the sensors. GENESI uses a wide range of electronic such as a power-scalable, high efficiency  DC-DC converters and low-drop regulators as well as different harvesting and power distribution strategies to maximise the provision and collection of energy under a large range of environmental conditions. The system also uses an effective power management strategy to compliment the energy harvester mechanism which improves the battery lifetime up to the theoretical limit.

Besides efficient energy harvesting methods, GENESI also emphasizes on reducing energy consumptions for communication and uses a selective activation scheme to dynamically schedule the activation of a set of passive and active sensors. A task allocation algorithm is used alongside the scheme to decide which sensors are to be activated and considers factors like node residual energy and field coverage. Choosing the best assignment of the available sensors to task proved a great challenge for the team since this problem involving nodes with radio triggering and harvesting capabilities has never been studied before.


\paragraph{Air Pollution} Continuing with the theme of deploying sensor networks in environments that more directly involve humans, air pollution has been a growing concern in congested urban cities as a result of industrialisation and heavy transportation. These concerns include poor air quality and visibility and long term damaging of human health \cite{libeliumAirPollution,wsnpollution}. Traditional methods of air quality monitoring involving quality control stations are expensive and provides low resolution sensing since monitoring stations are less densely deployed. Wireless sensor networks can be used as an urban monitoring system as it has the advantages of being small, easy to set up and inexpensive with real time monitoring capabilities. Sensors which have the ability to measure a wide range of meteorological data such as rainfall, wind speed, temperature, humidity and concentration of pollutants can be deployed in areas of high population density of vehicles and industrial areas. Collected data is transmitted back to the base station via a global system for mobile communication.

\citeauthor{fotuewsnpollution} in \cite{fotuewsnpollution} proposed a Wireless Mesh Network to establish internet connectivity and simultaneously measure air pollution in Sub-Saharan African cities. These cities experience the most problems due to the increase of industry, domestic waste and fuel combustion and that many chemical industries have been established in the central urban areas. However, with poor infrastructure and telecommunication links in Africa, it would be difficult to deploy a sensor network that will measure air pollution since some pollution areas are out of range of the communication service. In addition, some polluted areas lack the security measures or have industrial restrictions to deploy fixed sensors so they had to use mobile sensors which were attached on vehicles. 

The following are a number of challenges \citeauthor{fotuewsnpollution} faced when developing their solution.

\begin{itemize}
\item Reducing communication interference

High-gain antennas were used to reduce the probability of interference from non-city radio frequency emitters such as WiFi and access points. However, interference may also be caused from inter-device communication and was reduced by using multiple orthogonal channels. Each set of devices would use a unique channel to communicate with a different set of devices for example the mesh network would use a channel Cs with the sensors and channel Cg with the gateway.

\item Limiting number of request messages each sensor can receive from the data server.

The network uses an Ad-hoc On-Demand Distance Vector (AODV) routing protocol for efficient forwarding of data packets to the data server and makes use of a series of messages to communicate through the network and to discover and maintain data routes. A request message is propagated to the network when the data server requests for a specific data item. All the sensors receives the message and checks whether is it the destination or not. If not, it stores a copy of the message in its buffer containing previously broadcasted request messages and rebroadcasts it back to its neighbours. Once the destined sensor receives the message it sends back a reply message using the reverse path of the data server.

To prevent the sensors from overloading with request messages, each message is set with a Time-To-Live (TTL) value with discards the message when the value expires. This value is dependant on the size of the network where a larger network will have higher TTL value to ensure that the request messages reaches the defined destination.

\item Energy Usage

The wireless mesh network is energy efficient since the sensors are energy constrained. This uses an enhanced energy conserving routing protocol which employs a routing algorithm that aims to distribute data among maximally disjoint paths towards the sink so that the lifeline of the WSN service can be prolonged. It also allows the sensors to participate in carrying the global traffic over the network

\item Data transmission periods

To avoid high energy consumptions, data from the sensors to the data server were only transmitted at specific times of the day (non-real time). This allowed for energy and bandwidth optimization and thus increasing the network's lifetime on the long run.

\item Optimized placements of mesh router 

Since the network includes mobile sensors, there are chances that it can lose connectivity with the the fixed sensors. A challenge that the team faced was placing the mesh routers in such a way that they were always connected to the mobile and fixed sensors in order to receive the sensed data. This is crucial for network reliability and quality of service.
\end{itemize}


\subsubsection{Tools and Platforms}


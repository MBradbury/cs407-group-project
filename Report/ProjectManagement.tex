% !TeX root = Report.tex
\section{Project Management}

\subsection{Work Overview}

\subsubsection{Term 1}

\begin{center}
	\begin{longtable}{| l | p{7.5cm} | p{5cm} |}
	Week & Activities & Task Allocation\\
	\hline
	1 & \begin{enumerate}
			\item Met up with Supervisor and discussed project direction
			\item Decided to research what has been done and what we could do, before settling on main aims
			\item Investigated two OSes and their corresponding simulators TinyOS with TOSSIM and Contiki with Cooja
		\end{enumerate} &
	\begin{enumerate}
		\item[] Amit: 1, 2, 3
		\item[] Dan: 1, 2, 3
		\item[] Ivan: 1, 2, 3
		\item[] Joe: 1, 2, 3
		\item[] Matt: 1, 2, 3
		\item[] Tim: 1, 2, 3
	\end{enumerate}
	\\ \hline

	2 & \begin{enumerate}
			\item Research if Cooja can be extended through the use of plug-ins (with the aim of extending it to replay traffic logs)
			\item Research Clustering algorithms and find implementations
			\item Develop a temperature dissemination application to learn Contiki and Cooja
			\item Research TinyOS and TinyDB and see if they could be applied to predicate checking
			\item Investigate performance of different MAC protocols
			\item Investigate the feasibility of live monitoring of the network
			\item Investigate QoS and how it may be applied to real life WSN deployments
			\item Produce a literature review on chosen topic
		\end{enumerate} &
	\begin{enumerate}
		\item[] Amit: 1, 7, 8
		\item[] Dan: 5, 8
		\item[] Ivan: 6, 8
		\item[] Joe: 4, 8
		\item[] Matt: 3, 8
		\item[] Tim: 2, 8
	\end{enumerate}
	\\ \hline
	
	3 & \begin{enumerate}
			\item Research DICAS
			\item Research Send to Base
			\item Research Daicon
			\item Research DIDUCE
			\item Research H-SEND
			\item Research Sympathy
			\item Write Specification
		\end{enumerate} &
	\begin{enumerate}
		\item[] Amit: 2, 7
		\item[] Dan: 4, 7
		\item[] Ivan: 3, 7
		\item[] Joe: 1, 7
		\item[] Matt: 6, 7
		\item[] Tim: 5, 7
	\end{enumerate}
	\\ \hline
	
	
	4 & \begin{enumerate}
			\item Finish Specification
			\item Work out how to get MSP430-GCC and the Cooja working on Joshua
			\item Begin developing aggregation tree example
		\end{enumerate} &
	\begin{enumerate}
		\item[] Amit: 1, 2
		\item[] Dan: 1
		\item[] Ivan: 1
		\item[] Joe: 1
		\item[] Matt: 1, 3
		\item[] Tim: 1
	\end{enumerate}
	\\ \hline
	
	
	5 & \begin{enumerate}
			\item Get used to developing in C and learn how to use the Contiki libraries
			\item Finish developing aggregation tree
			\item Start implementing clustering algorithms
			\item Start implementing H-SEND
		\end{enumerate} &
	\begin{enumerate}
		\item[] Amit: 1, 4
		\item[] Dan: 1, 3
		\item[] Ivan: 1, 3
		\item[] Joe: 1, 4
		\item[] Matt: 1, 2
		\item[] Tim: 1
	\end{enumerate}
	\\ \hline
	
	
	6 & \begin{enumerate}
			\item Start Implementing GUI
			\item Implementing clustering algorithms (Regular and Hierarchical)
			\item Implement H-SEND
			\item Implement a local predicate checker and reporter
		\end{enumerate} &
	\begin{enumerate}
		\item[] Amit: 3
		\item[] Dan: 2
		\item[] Ivan: 2
		\item[] Joe: 3
		\item[] Matt: 4, 2
		\item[] Tim:\subsubsection{Term 1} 1
	\end{enumerate}
	\\ \hline
	
	7 & \begin{enumerate}
			\item Library-ify Cluster implementations
			\item Implement LEACH  using existing implementations as a base
			\item Work out how to interface the GUI and the base station
			\item Continue developing GUI
			\item Get in contact with Sain Saginbekov and sort out when we can use sensor nodes
			\item Contact Victor Sanchez and inform him when we are using sensor nodes
			\item Implement multi-hop predicate checking
			\item Slim down eLua enough for us to run it on the sensor nodes, to allow for dynamic predicate specification
			\item Research and start implementing network traffic logging and reporting
		\end{enumerate} &
	\begin{enumerate}
		\item[] Amit: 7
		\item[] Dan: 1, 2
		\item[] Ivan: 9
		\item[] Joe: 5, 6, 7
		\item[] Matt: 8
		\item[] Tim: 3, 4
	\end{enumerate}
	\\ \hline
	
	8 & \begin{enumerate}
			\item Start work on poster with the aim of finishing it by the end of week 9
			\item Detect and report neighbours for display in the visualisation tool, aim to be able to work in mobile networks, with gained and lost neighbours
			\item Log network data and report it to the network (for visualisation tool to replay traffic)
			\item Finish predicate language parser and code generation
			\item Tidy up n-hop HSEND
			\item Integrate n-hop HSEND with predicate language virtual machine
			\item Experiment with interference affecting node communications
		\end{enumerate} &
	\begin{enumerate}
		\item[] Amit: 1, 5, 6
		\item[] Dan: 1, 3, (possibly 7)
		\item[] Ivan: 1, 2, (possibly 7)
		\item[] Joe: 1, 2, (possibly 7)
		\item[] Matt: 1, 4, 6
		\item[] Tim: 1, 4
	\end{enumerate}
	\\ \hline

	9 & \begin{enumerate}
			\item Create poster for presentation
		\end{enumerate} &
	\begin{enumerate}
		\item[] Amit: 1
		\item[] Dan: 1
		\item[] Ivan: 1
		\item[] Joe: 1
		\item[] Matt: 1
		\item[] Tim: 1
	\end{enumerate}
	\\ \hline

	10+ & \begin{enumerate}
			\item Code generation for predicate language
			\item Network visualisation (node neighbours)
			\item GUI to Node interface
			\item Message logging
			\item Neighbour Reporting
			\item Turn existing code into libraries
			\item Make existing code handle generic data
			\item Integrate predicate evaluating VM and neighbour data retrieval
			\item Write up notes on work done this term
		\end{enumerate} &
	\begin{enumerate}
		\item[] Amit: 6, 7, 8, 9
		\item[] Dan: 4, 5, 9
		\item[] Ivan: 3, 5, 9
		\item[] Joe: 6, 7, 8, 9
		\item[] Matt: 6, 7, 8, 9
		\item[] Tim: 1, 2, 9
	\end{enumerate}
	\\ \hline
	
	\end{longtable}
\end{center}

\subsubsection{Term 2}

\begin{center}
	\begin{longtable}{| l | p{7.5cm} | p{5cm} |}
	\hline
	Week & Activities & Task Allocation\\
	\hline
	1 & \begin{enumerate}
		\item Continue network visualisation
		\item Create a test suite for network visualisation
		\item Interface Base station node with desktop application
		\item Develop neighbour detection and reporting to sink
		\item Integrate predicate evaluating, predicate VM and neighbour data retrieval
		\item Finish parser and compiler of predicate language
		\end{enumerate} &
	\begin{enumerate}
		\item[] Amit: 5
		\item[] Dan: 4
		\item[] Ivan: 3
		\item[] Joe: 5
		\item[] Matt: 5
		\item[] Tim: 1, 2, 6
	\end{enumerate}
	\\ \hline
	
	\end{longtable}
\end{center}


\subsection{Role Allocation}

We decided to allocate roles in the second week after we had the first week to perform research into the problem and find out what has been done. The following were how we assigned roles, although we intend for these to be flexible:

\begin{table}[H]
\centering
	\begin{tabular}{| c | c |}
		\hline
		Name & Role\\
		\hline
		Matthew Bradbury & Group Leader\\
		Tim Law & Developer and Researcher\\
		Ivan Leong & Developer and Tester\\
		Daniel Robertson & Project Manager\\
		Amit Shah & Technical Leader\\
		Joe Yarnall & Developer and Tester\\
		\hline
	\end{tabular}
\end{table}


\subsection{Schedule}


\begin{table}[H]
	\centering
	\begin{tabular}{| l | l | l | l | l | l | l |}
	\hline
	Task Description & \multicolumn{6}{l|}{Time Allocated (Weeks)}\\
	~ & Amit & Dan & Ivan & Joe & Matt & Tim \\
	\hline
	\hline
	\multicolumn{7}{|l|}{\textbf{Term 1} - Developing for application predicate checking} \\
	\hline


	Research around the Problem & 2 & 2 & 2 & 2 & 2 & 2\\
	Writing Specification & 1 & 1 & 1 & 1 & 1 & 1\\
	H-SEND Implementation & 3.5 & ~ & ~ & 3.5 & ~ & ~\\
	``Send to Base'' Implementation & ~ & ~ & ~ & ~ & 2 & 1.5\\
	Clustering Implementation & ~ & 3.5 & 2 & ~ & ~ & ~\\
	Aggregation Tree Implementation & ~ & ~ & ~ & ~ & 1.5 & ~\\
	Develop Visualisation Tool & ~ & ~ & 1.5 & ~ & ~ & 2\\
	Testing and Adapting to Physical Nodes & 2 & 2 & 2 & 2 & 2 & 2\\
	Message Logging & ~ & ~ & ~ & 1 & ~ & ~\\
	Poster Creation and Presentation preparation & 1.5 & 1.5 & 1.5 & 1.5 & 1.5 & 1.5\\

	\hline
	\hline
	\multicolumn{7}{|l|}{\textbf{Term 2} - Developing for network predicate checking and where the predicate is checked} \\
	\hline
	
	Additional Research & 1 & 1 & 1 & 1 & 1 & 1\\
	Improving Dynamic Predicate Specification & 2 & 2 & ~ & ~ & 2 & 2\\
	Develop Visualisation Tool & 2 & ~ & 2 & 2 & ~ & 2\\
	Modify Algorithms to Selectively Evaluate Predicates & ~ & 2 & 2 & 2 & 2 & ~\\
	Performance Testing & 1.5 & 1.5 & 1.5 & 1.5 & 1.5 & 1.5\\
	Testing and Adapting to Physical Nodes & 1.5 & 1.5 & 1.5 & 1.5 & 1.5 & 1.5\\
	Report Writing & 2 & 2 & 2 & 2 & 2 & 2\\
	
	\hline
	
	\end{tabular}
\end{table}

\subsection{Time Spent}

\begin{table}[H]
	\centering
	\begin{tabular}{| l | l | l | l | l | l | l |}
	\hline
	Task Description & \multicolumn{6}{l|}{Time Allocated (Weeks)}\\
	~ & Amit & Dan & Ivan & Joe & Matt & Tim \\
	\hline
	\hline
	\multicolumn{7}{|l|}{\textbf{Term 1} - Developing for application predicate checking} \\
	\hline


	Research around the Problem & 2 & 2 & 2 & 2 & 2 & 2\\
	Writing Specification & 1.5 & 1.5 & 1.5 & 1.5 & 1.5 & 1.5\\
	H-SEND Implementation & ~ & ~ & ~ & ~ & 2 & ~\\
	``Send to Base'' Implementation & ~ & ~ & ~ & ~ & 2 & ~\\
	Clustering Implementation & ~ & ~ & ~ & 3 & 0.5 & ~\\
	Aggregation Tree Implementation & ~ & ~ & ~ & ~ & 1 & ~\\
	Develop Visualisation Tool & ~ & ~ & ~ & ~ & 0 & ~\\
	Testing and Adapting to Physical Nodes & ~ & ~ & ~ & ~ & 0.5 & ~\\
	Message Logging & ~ & ~ & ~ & 1 & ~ & ~\\
	Poster Creation and Presentation preparation & ~ & ~ & ~ & ~ & 0.5 & ~\\

	\hline
	\hline
	\multicolumn{7}{|l|}{\textbf{Term 2} - Developing for network predicate checking and where the predicate is checked} \\
	\hline
	
	Additional Research & ~ & ~ & ~ & ~ & ~ & ~\\
	Improving Dynamic Predicate Specification & ~ & ~ & ~ & ~ & ~ & ~\\
	Develop Visualisation Tool & ~ & ~ & ~ & ~ & ~ & ~\\
	Modify Algorithms to Selectively Evaluate Predicates & ~ & ~ & ~ & ~ & ~ & ~\\
	Performance Testing & ~ & ~ & ~ & ~ & ~ & ~\\
	Testing and Adapting to Physical Nodes & ~ & ~ & ~ & ~ & ~ & ~\\
	Report Writing & ~ & ~ & ~ & ~ & ~ & ~\\
	
	\hline
	
	\end{tabular}
\end{table}



\subsection{Working Concurrently}

We signed up for a Git repository on BitBucket \cite{bitbucket} where we plan to commit all the work we produce. We initially had an issue that free private repositories hosted on BitBucket have a maximum of 5 participants, whereas we had 6 group members. Fortunately when new users sign up to the services from an invite, the person that sends the invite gets additional capacity. This meant that the person who created the repository ended up with enough capacity for all members to access the account.

\subsection{Group Members Without Internet}
Unfortunately two of our group members were without internet for the first 3 weeks of term. This was an issue because they were unable to just work in the DCS labs because the computers there didn't have the software required (such as VirtualBox or the WSN simulators). To work around this, those two members were given tasks that could be accomplished with their own machines and without internet. Once they obtained internet they were given tasks, that access allowed them to accomplish.

\subsection{Society Exec Clashes}
Two members of our team are on the exec of Warwick Societies, this means that at certain parts of the year they were in high demand for those jobs. For instance during the first few weeks of term during Freshers' Fortnight, the exec members were required to be involved in many of their events that they were running. However, after these first weeks the demand on their time decreased and made balancing the time been the demands of this project and their exec roles much easier.


% !TeX root = Report.tex
\section{Evaluation and Conclusions}

To conclude we have found this a difficult project to complete. Mostly this was due to the fact that a number of the libraries that we expected to be available were not, our unfamiliarity with Contiki and the initial lack of direction for the project. We have learnt that it is much better to have a very well defined goal to being with, even in a research project such as this.

As part of this project we have produced four different predicate evaluation libraries that were analysed to find the relative performances between evaluating predicates in-network or at a sink node after gathering the network's data. We also investigated under what circumstances should a node's data be sent, when it changes or periodically, and how to send that data. In order to aid system administrators of sensor networks we developed a GUI that could communicate with the network, this GUI was used to send predicates into the network and visually display the network's configuration and predicate failure responses to the user. To evaluate predicates a virtual machine and scripting language were developed so that new predicates could be deployed and old ones updated or removed. Our aim was to eliminate the need to redeploy the firmware across the networks.

From our results some libraries perform well in certain situations and other libraries in other situations, which means that it is up to the developers of systems to choose which to deploy. We believe that there is much room for improvement in the number of messages sent and the number of failure responses received. Improving the percentage of predicates evaluated successfully may be harder due to the time delays between data being generated and sent and the predicates being evaluated.

As a group we've had to deal with the challenges of organisation, making sure everybody knew what they had to do, and when it was needed for completion. This proved to be initially complex working around the schedules of each of the team members, but tasks were modularised to help negate this issue. Making sure each team member was contactable and was kept up-to-date with the current progress of the project ensured that everything was run as smoothly as possible. 

Overall, we have learnt a lot as a group in terms of undertaking projects and how to develop for unreliable distributed systems such as wireless sensor networks. Our initial aims were to explore tools assisting in the debugging of WSNs, and our final systems makes great progress towards those goals. The tools have been shown to consistently evaluate predicates across different network sizes, subject to traditional drawbacks within WSNs (such as message loss and collisions). The visualisation tool itself allows the adjustment of these predicates, and the display of the current network status for a live deployed network, and not just a simulated network (such as COOJA). There is still a large amount of work that could be done, including expanding the abilities of the virtual machine, and the visualisation tool's display properties. We hope that some of the work here can be contributed back to the open source community where it will help others to develop applications with Contiki.


\section{Predicate Evaluation}

\subsection{Predicate Scripting Language}

\subsubsection{Required Features}

\begin{enumerate}
	\item Numerical support (Integer and Floating point) ($+$ $-$ $\times$ $/$ $=$ $\neq$ $<$ $\leq$ $>$ $\geq$)
	\item Set support (Creation and operating on) ($\cup$ $\cap$ $\setminus$ $\{\}$ $\forall$ $\exists$)
	\item Limited function support (eg. $Neighbourhood(N, H)$)
	\item Limited struct support (eg. $N.Temperature$)
	\item Ability to specify predicate target(s) (Single node, multiple nodes or entire network)
	\item Ability to get the current node ($this$)
	\item Ability to get sensor information on a node (temperature, humidity, \ldots)
	\item Ability to get node information (Distance to Sink, \ldots)
\end{enumerate}

\subsubsection{Non-Required Features}

There are lots of features that are usually provided by scripting languages that would cause undesirable effects such as inflating the binary size.

\begin{enumerate}
	\item Strings
	\item Complicated data structures (Lists, Arrays, Dictionaries, \ldots)
	\item Library features (Custom libraries or built-in libraries)
	\item Interactive mode (terminals)
	\item IO
\end{enumerate}


\subsubsection{Evaluation Methods}

\begin{enumerate}
	\item Send scripting language to node, they interpret and evaluate it
	\item Compile predicate on base station, compress and send to node(s) for evaluation
\end{enumerate}

\subsubsection{Scripting Languages Considered}

\begin{enumerate}
	\item eLua \url{http://www.eluaproject.net/}
	\item SCript \url{http://www.sics.se/~adam/dunkels06lowoverhead.pdf} \cite{dunkels06lowoverhead}
	\item wren \url{https://github.com/darius/wren}
	\item Write the predicate in C, compile it to machine code and send it to the nodes to be evaluated
\end{enumerate}


\subsubsection{Getting and Waiting for Neighbourhood Data}

\begin{enumerate}
	\item If all neighbours are known then 1-hop predicates can wait for each node to send data. This has extra memory and communication requirements
	\item If we do not know who our n-hop neighbours are then we can just set a time out to wait for results
	\begin{enumerate}
		\item If we do not receive all items what should be done?
		\item Should we report that the predicate is true, even though we didn't receive all the information?
	\end{enumerate}
	\item What data should be sent from surrounding nodes and how could we access it?
\end{enumerate}



\subsection{Where to Evaluate Predicates}

Given the problem of assigning slots in a TDMA MAC protocol, one would wish to ensure that no node within two hops of any given node has the same slot. This raises a question of where this predicate should be evaluated to ensure minimum energy usage in its evaluation.

The first predicate we might consider is the one where for ever node, we get the 2-hop neighbourhood and check that none of those nodes have the same slot as the initial node. This means that we would need to send:
\begin{enumerate}
	\item A message from the base station to the node asking it to evaluate the predicate
	\item A message from the node to each of its 2-hop neighbours
	\item Each 2-hop neighbour needs to send a message back to the node
	\item The node would need to wait for its neighbours to send the messages, then evaluate the predicate. This would imply that the node would need to know who is in its two hop neighbourhood
	\item The node would need to report to the base station the result of the predicate
\end{enumerate}

So we would need at best $2\Delta_{sink} + 2|Neighbours(2, n)|$ messages to evaluate this predicate for one node.

\begin{align}
\label{eq:2-hop-slot-pred}
& \hspace{3em}	\forall n \in Nodes \cdot \\
& \hspace{6em}		\forall n' \in Neighbours(2, n) \cdot \\
& \hspace{9em}			(n.slot \neq n'.slot)
\end{align}


The second predicate we could consider is the one where every node checks their 1-hop neighbourhood for slot collisions. Here we model the 2-hop nature of the predicate differently, because instead of checking on the node that we want to check for slot collisions, we check on the node between two nodes that might have slot collisions.

\begin{enumerate}
	\item A message from the base station to the node adjacent to that we asking to check slot collisions
	\item A message from the node to each of its 1-hop neighbours
	\item Each 1-hop neighbour needs to send a message back to the node
	\item The node would need to wait for its neighbours to send the messages, then evaluate the predicate. This would imply that the node would need to know who is in its two hop neighbourhood
	\item The node would need to report to the base station the result of the predicate
\end{enumerate}

So we would need at best $2(\Delta_{sink} + 1) + 2|Neighbours(1, n)|$ messages to evaluate this predicate for one node.

\begin{align}
\label{eq:1-hop-slot-pred}
&				\forall n \in Nodes \cdot \\
& \hspace{3em}		\forall n' \in Neighbours(1, n) \cup \{n\} \cdot \\
& \hspace{6em}			\forall n'' \in Neighbours(1, n) \cup \{n\} \cdot \\
& \hspace{9em}				(n' \not= n'' \land n'.slot \neq n''.slot)
\end{align}

Overall we can assume that $|Neighbours(1, n)| \leq |Neighbours(2, n)|$, so checking 1-hop neighbours would in general require fewer message. Also we can assume that when checking these predicates, we would make sure that every node is checked (as the predicates are defined). This means that the first predicate would duplicate the checks: checking node $n$ for a collision with $n'$ and when $n'$ is asked to check the predicate it will query $n$.

However, if we were not asking the entire network to check for this property, then the first predicate would behave better. This is because the second predicate would have to be called for every 1-hop neighbour of the intended node. However, as it is assumed that the entire network would be asked if this predicate holds, then the second predicate is better.


\documentclass[a4paper]{article}

\usepackage[top=2cm, bottom=2cm, left=3cm, right=3cm]{geometry}

\usepackage{float}

\usepackage{multirow}

\usepackage[hyphens]{url}
\usepackage{hyperref}

\usepackage{appendix}
\usepackage[numbers]{natbib}

\usepackage{graphicx}

\usepackage[parfill]{parskip}
\setlength{\parindent}{0pt}
\setlength{\parskip}{\baselineskip}

\newcommand{\mytilde}{\raise.17ex\hbox{$\scriptstyle\mathtt{\sim}$} }

\title{CS407 - Project Specification:\\
Towards Debugging Wireless Sensor Network Applications}
\date{25th October 2012}
\author{
	Matthew Bradbury (0921660) \and
	Tim Law (0918647) \and
	Ivan Leong (0830934) \and
	Daniel Robertson (0910210) \and
	Amit Shah (0904778) \and
	Joe Yarnall (0905247)
}

\begin{document}

\maketitle

%No numbering on first page
\pagestyle{empty}
\thispagestyle{empty}

\newpage

\pagestyle{plain}
\setcounter{page}{1}

\tableofcontents
\clearpage


\section{Introduction}

The emergence of Wireless Sensor Networks (WSNs) and the decrease in their cost and size has made it feasible to use them to solve certain types of problems. To solve these problems software will need to be developed for the WSN hardware, and like all software this software will contain bugs \cite{5010224}. As sensor networks are a distributed system, this increases the classes of bugs that software can suffer from \cite{5010224}, such as dead-lock, live-lock and other concurrency issues. To aid in developing reliable software for WSNs, there is thus need for tools that can assist in detecting and reporting these faults, so a developer can correct them.

\section{Motivation}

The need for debugging has been around for as long as there has been hardware that performs a task. With a famous folklore example being the case where the term debugging was coined, where a moth was picked out of a crucial relay \cite{shapiro1987etymology}. However, nowadays debugging searches for much more complicated problems than moths in machines.

When you have a sequential program it is easy to insert breakpoints or debug messages that allow you to inspect the state of the program at that given point of the execution. With distributed systems, you have the same issues that sequential programs have, but you also have to deal with the interaction of each of the different parts of the system. These interactions can be non-deterministic \cite{liu2007wids} and make inspection of the system's state at any given point much harder \cite{?}. These interactions can lead to many types of bugs that a sequential program would not encounter, such as deadlock \cite{singhal1989deadlock}, inconsistent states \cite{?} and the need to handle failures in other disconnected parts of the system \cite{?}. While there exist algorithms to solve some of these problems (for instance TCP provides stronger guarantees of message deliver than alternative protocols such as UDP \cite[p.~532]{Tanenbaum:2002:CN:572404}), there are some errors that the software developer will need to ensure they do not make when writing the software. As detecting these issues is difficult and so is reporting them, tools need to be made available to the developers, so they can focus on developing their application rather than developing debugging tools to try to fix their application.

Wireless Sensor Networks are a class of distributed system, with a special property that they are energy constrained \cite{6023235}.  As they are a distributed system, the same distributed bugs can occur in software developed for them, so there is a need for tools to assist with this kind of debugging. However, as they are energy constrained developers will attempt to make certain optimisations that may not have been required with other types of distributed systems. This means any tool that is specific to them will need to be able to detect these kinds of bugs and relay them to the developer accurately.


\section{Project Description}

\subsection{Aims and Objectives}
\begin{itemize}
	\item An application that can check to see if for a certain subsection of the network a given predicate holds.
	\item A visualisation tool for logged network traffic.
	\item Implementations of various distributed algorithms to test the visualizer and the predicate checker.
\end{itemize}

\subsection{Visualisation Tool}
\begin{itemize}
	\item Visualisation of the sensor network with node positioning and connections.
	\item Visualisation of active clusters within the sensor network.
	\item Visualisation of messages flowing through the network, along with node logs of the information processed.
\end{itemize}

\subsection{Predicate Checker}
\begin{itemize}
	\item Functionality to query nodes to see if a given predicate holds.
	\item Visualisation of nodes that are breaking a given predicate during runtime.
\end{itemize}

\subsection{Required Resources}
\begin{itemize}
	\item Wireless Sensor Nodes and related interface hardware
	\item Operating System and Simulator for the Wireless Sensor Nodes
	\item Laptop computers to act as mobile base stations, and for development
	\item Access to academic papers that detail the previous work
	\item A DVCS where we can store source code for the project
	\item A simulator and development environment that will allow us to test code locally, then deploy to the actual hardware. There is a choice between Contiki \cite{23839452} and TinyOS \cite{levis2003tossim}.
\end{itemize}

\section{Project Management}

\subsection{Management}

As this is a group project there are additional challenges to face compared to a solo project. Therefore, we plan to use the knowledge and experience gained in the module IB382 Project Management \cite{IB382} as well as our own experience to avoid common problems that could prevent us working effectively. We also plan to refer to the Project Management textbook for additional resources \cite{PMTextBook}.

As part of the management we will make sure we meet up at least once a week to discuss our current progress, and allocate tasks for the next week. We also intent to stay in contact and discuss our progress and any resources we find in between these meetings.

We will also be assigning roles, in order to keep the group organised. The following table shows how we assigned roles, although we intend for these to be flexible:

\begin{table}[H]
\centering
	\begin{tabular}{| c | c | c |}
		\hline
		Name & Role & Role Description\\
		\hline
		Matthew Bradbury & Group Leader & ~ \\
		Tim Law & nil & ~ \\
		Ivan Leong & nil & ~ \\
		Daniel Robertson & nil & ~ \\
		Amit Shah & nil & ~ \\
		Joe Yarnall & nil & ~ \\
		\hline
	\end{tabular}
\end{table}



\subsection{Development Methodologies}
In our first meeting with our project supervisor, we were given the objective to develop a wireless sensor network (WSN) with the new set of sensors that was purchased by the department of Computer Science. He proposed several ideas that we could use to implement the system (e.g. routing protocols and algorithms) and a few topics on how we can investigate the problems associated with WSN such as global predicate detection and network debugging.

For our project, we plan to follow a spiral evolutionary process model since we are expecting many changes in our system, for example trying to implement the best algorithm for predicate checking and experimenting with different routing protocols. Although we are clear on the objectives, we have yet to define the detailed requirements for the functions and features of the system. Hence, we aim to develop our system in a series of evolutionary steps that will accommodate changes throughout the development. TBC.



\subsection{Schedule}

\subsubsection{Term 1}

During the first term, we plan to do our initial research around the subject to find out what has been done and what is implementable for us. We then plan to implement some of the algorithms to detect predicate failures we researched in a simulator. Once the algorithms are working in a simulator we will deploy them to the motes. We expect difficulty deploying to the motes due to: i) us needing to learn how to do it, ii) the environment being slightly different to the simulators and iii) debugging our algorithm will be much harder on the hardware than in a simulator. By the end of week 9 we should have a basic implementation on stationary nodes.

\begin{table}[H]
	\centering
	\begin{tabular}{| l | c |}
	\hline
	\textbf{Task} & \textbf{Weeks}\\
	\hline
	Initial Research & 1-3\\
	\hline
	Specification Writing & 3\\
	\hline
	Algorithm Implementation & 4-6\\
	\hline
	Testing & 7\\
	\hline
	Deploying to Physical Hardware & 8-9\\
	\hline
	Poster Creation and Presentation Preparation & 10\\
	\hline
	\end{tabular}
\end{table}

\subsubsection{Term 2}

We do not expect to fully finish development for stationary nodes during the first term, so we will continue development during the second term. We also aim to research and implement different algorithms focusing on different aspects of wireless sensor networks (such as node mobility). Again a lot of time is devoted to testing and deployment due to the difficulties involved.

\begin{table}[H]
	\centering
	\begin{tabular}{| l | c |}
	\hline
	\textbf{Task} & \textbf{Weeks}\\
	\hline
	Further Research & 1\\
	\hline
	Algorithm Implementation & 2-4\\
	\hline
	Testing & 5-6\\
	\hline
	Deploying to Physical Hardware & 7-8\\
	\hline
	Report Writing & 9-10\\
	\hline
	\end{tabular}
\end{table}



\section{Potential Risks}

Throughout our project we expect to encounter difficulties and challenges. The following are the risks that we think may be encountered during the project and ways that these risks could be mitigated.


\subsection{Risks}
\begin{itemize}
	\item Needing to talk to our project supervisor (Arshad), even though he is on sabbatical.
	\item Project drifting in the wrong direction due to uncertainty.
	\item Availability of hardware, because it is shared by the department.
	\item Lack of features in the simulator and/or wireless sensor network environment.
	\item Problems that we have not predicted here.
	\item Team members being unavailable to do work (eg. due to illness).
\end{itemize}

\subsection{Risk Mitigation}
\begin{itemize}
	\item Even though Arshad is on sabbatical he should be in the department 3 days a week. Also there are other departmental staff that can assist us if Arshad is not available.
	\item If our project is not heading in the right direction, we will seek guidance from our supervisor who should be able to guide us onto the right track.
	\item Additionally to stop project drift we will have weekly meetings to assess our progress and set clear goals for each member of the group.
	\item If our development environment lacks features we can attempt to work around those features, or implement them (if they are very desirable) and contribute back to the open source community.
	\item For any other problem, we will use all the support that the department provides. We will make sure that the issue is discussed within the group and we will try to resolve it ourselves. if we find we cannot, then we will seek additional help.
	\item Each member will make sure that what they are doing is well documented, so that other members can easily pick up the work. Each member of the team will also ensure their work is regularly checked in to the git repository.
\end{itemize}


\section{Legal Issues}
In some cases there is the potential for copyright infringement when dealing with certain algorithms, this however itself is unlikely, as we will be focusing on published public algorithms. We will also not be looking to turn a profit from the use of these algorithms, but to use them for demonstrations of our developed tools.

There is also an issue of using pieces of open source software, we will make sure that the use of any such software is documented, with the appropiate acknowledgements given.

\section{Conclusion}

For our project we aim to create tools that aid in the debugging of Wireless Sensor Network applications. We plan to implement various predicate detection algorithms and have the network report to a base station in the case that these predicates are violated. We also plan to create a new tool that can control submitting new predicates and visualise the network state, including any predicate failures.

\clearpage

\appendixpage
\addappheadtotoc
\appendix


\section{References}
\renewcommand{\refname}{\vspace{-1cm}}
\bibliographystyle{myplainnat}
\bibliography{../References/references}

\end{document}

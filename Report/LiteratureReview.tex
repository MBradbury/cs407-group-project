% !TeX root = Report.tex
\section{Literature Review}

\subsection{Routing in Sensor Networks}

\subsubsection{Flooding}



\subsection{Simulating Sensor Networks}

\subsubsection{TinyOS}

\subsubsection{Contiki}
% Cite basically all the papers at: http://www.contiki-os.org/support.html

\subsubsection{NS2}

\subsubsection{JProwler}

\subsubsection{Simulating Energy Consumption}
\cite{Shnayder04}


\subsection{Sensor Network Platforms}

\subsubsection{MICA}
\cite{Mica2002}

\subsubsection{Sky}


\subsection{Practical Experience in Sensor Networks}

\subsubsection{Air Pollution}
\cite{libeliumAirPollution}
\cite{wsnpollution}

\subsubsection{Forest Fires}
\cite{libeliumForestFires}
\cite{FireWxNet}

\subsubsection{Great Duck Island}
\cite{SzewczykPMC04}

\subsubsection{Volcano Monitoring}
\cite{Werner-Allen:2006:FYV:1298455.1298491} Claimed: Ivan

Active volcanoes need to be monitored to predict the likihood of future eruptions so that early-warning signs can be issued for the evacuation of habitants near the volcano. Today, volcanoists use wired arrays of sensors such as seismometers and acoutic microphones to collect seismic and infrared (low-frequency acoustic) signals and determining a wide range of factors such as sources of volcanic eruptions, interior structure of a volcano and differentiating eruption signals from noise. A typical study can indlue the placement of several stations around the volcano where each contains a low distribution of wired sensors that collects data to a hard drive. Data is then collected manually in a possibly inconvenient location.

To address the issue of high power consumption of these wired sensors, Werner-Allen Et Al. \cite{?} introducted the deployment of embeddded wireless sensor networks which consists of low-power nodes with minimum CPU, memory and wireless communication capabilities.


\subsubsection{Structural Monitoring}
\cite{5508230}

\subsubsection{Development of Applications}
\cite{Fagerstrom:1988:DTD:55823.55833}


\subsubsection{Power Considerations}


\subsection{Predicates}

Before we can consider what predicates we can try to check for, or how we should check those predicates, we first need to understand what we are detecting and why. In dependable systems we need to consider the fault-error-failure cycle. This cycle says that a \emph{fault} once caused by either some external influence (e.g. radiation leading to bit-flips in memory \cite{?}) or internal influence (e.g. code bugs) will lead to an \emph{error}, this is the \emph{activation} step. An error is the manifestation of the fault (e.g. memory holding the incorrect value). An error then leads to a \emph{failure} in the step called \emph{propagation}, the failure of the system is an observable deviation from the system's specification (e.g. allowing doors to be opened that should remain closed). It is not always the case the faults lead to errors, or errors lead to failures, sometimes multiple faults or errors are respectively required to cause a single error or failure. \cite{1335465}

There is a choice of what should be measured and checked in predicates, should faults, errors or failures be measured? \textbf{TODO}


\subsubsection{GPD}

\cite{345831}
\cite{277788}
\cite{553309}


\subsection{Predicate Detection in Distributed Systems}

\subsubsection{NodeMD}
\cite{NodeMD}


\subsubsection{Sympathy}

One of the projects that is summarised by \citeauthor{herbert2007adaptive} in \cite{herbert2007adaptive} is a method for identifying and localizing failures called Sympathy \cite{ramanathan2005sympathy}. Sympathy is intended to be run either in pre- or post-deployment environments where it collects data from distributed nodes at a sink. When insufficient data is received it is taken to imply that there exists a problem (insufficient data is component defined). The idea is that by monitoring data (both actively and passively) between components the system can identify what kind of failure occurred in a certain area of the network. Both of which are very useful when trying to debug a failure.

It does, however, have some downsides. The first is that there is assumed to be no traffic and thus no application traffic or network congestion. These are real issues especially when applying this kind of debugging to a high throughput sensor network. There are also a number of spurious failure notification, which the authors are working on reducing, by applying a Bayes engine.

\subsection {DICAS: Detection, Diagnosis and Isolation of Control Attacks in Sensor Networks}

Detection, Diagnosis and Isolation of Control Attacks in Sensor Networks (DICAS) \cite{dicaspaper} is a lightweight distributed protocol for Wireless Sensor Networks that mitigates the effects of a wide-range of control traffic attacks. DICAS does this by utilizing a unique property of a wireless sensor network, this is that each node is able to monitor a part of their neighbours network traffic. Using DICAS the authors were able to create LSR a lightweight secure routing algorithm for Wireless Sensor Networks.

In DICAS nodes maintain a data structure of their first hop neighbours for local monitoring to detect malicious nodes and in local response to isolate these nodes. When a node is deployed it finds and authenticates all of its 1 hop neighbours using a pairwise shared key. These neighbours communicate their neighbours, 2 hop neighbours to the original node, and their own commitment key, which is generated using a random seed, to the newly deployed node. Each node will then have knowledge of all of its 1 hop neighbours as well as their commitment keys and all of its 2 hop neighbours.

Using this knowledge the DICAS protocol can enact a collaborative detection strategy where every node monitors the traffic going in and out of it's neighbours. Each node that is within transmission range of both the sending(X) and receiving(A) nodes of a packet are considered guard nodes of A over the link from X to A. These nodes maintain a watch buffer of packets sent from X to A, the duration and information stored are determined by the attack type under consideration by the system. Each guard node maintains a malicious counter for each link it is monitoring, if A drops, delays, changes or fabricates a packet from X then the guard node will increase its malicious counter. If the malicious counter exceeds a predefined threshold the node is considered to be malicious and is removed from the neighbours list. The guard node propagates this knowledge to all the other nodes in its neighbour list. When another node receives enough authenticated alerts about the malicious node it is excluded from its neighbours list. Once all 1 hop nodes have excluded the malicious node then it is effectively isolated from the system and all packets from or to that node are ignored.      

The authors tested the DICAS algorithm against 5 sets of attacks, these are:
\begin{enumerate}

\item Route Traffic Manipulation

\item ID Spoofing and Sybil Attacks

\item Wormhole Attacks

\item Sinkhole

\item Rushing Attack

\end{enumerate}

Additionally, the paper describes a cost analysis on the DICAS algorithm, these are the results: 
\begin{itemize}
\item Memory Overhead:

The memory overhead is the most pressing of the overheads created by the DICAS algorithm. The algorithm must store several data structures on each node: a neighbours list, a watch buffer, a commitment key list and the alert list. These structures are variable on sizes dependent on the number of nodes in the network, the network layout and the MAC layer delay for acquiring a channel. Any implementation must consider the memory cost of the algorithm seriously.

\item Computation Overhead:

With regards to computation they found that each packet received or sent required: one lookup for the current source and destination in the neighbour list, for an incoming packet - adding an entry to the watch buffer or for an outgoing packet - deleting an entry from the watch buffer. Since the size of the watch buffer and the neighbour list structure are relatively small, the computation time required for these operations is negligible.

\item Bandwith Overhead:

When considering bandwidth the overhead was primarily gained in 2 conditions: after node deployment when a node is populating its neighbour list and during a wormhole attack detection where a node is informing its neighbours of the malicious node. However, these cases make up a negligible fraction of the total network traffic over the lifetime of a wireless sensor network.
\end{itemize}

\subsection{H-SEND: Hierachichal SEnsor Network Debugging}

H-SEND is a framework for detecting faults in WSNs, designed to minimise energy consumption. It differs from related algorithms by being capable of handling very large WSNs. \cite{?}

There are 4 main steps involved in H-SEND:

\begin{enumerate}
	\item Developer specifies invariants when writing the software
	\item Invariant checking code is (semi--)automatically inserted during compilation
	\item If an invariant is violated at runtime, actions are taken (such as increased logging frequency, or an error message to the base station)
	\item Developer can use the information to fix the software, and then upload the patched version
\end{enumerate}

There are a variety of different types of invariants that can be specified, typically characterised by the following three dichotomies:

\begin{itemize}
	\item Local vs. Multi--node invariants
	
	Invariants dependent on state on a single node are called local, as they don't require and messaging to check. If the state of multiple nodes is required for checking, the invariant is said to be multi--node.

	\item Stateless vs. Stateful invariants
	
	An invariant is stateless is it doesn't depend on a node's execution state, and stateful otherwise.

	\item Compile--time vs. Run--time invariants

	Compile--time invariants are those involving comparisons against values that do not change, whereas runtime invariants are more flexible. Runtime invariants can compare against spatial and temporal trends ---  the state of surrounding nodes, and previous states respectively.
\end{itemize}

A grammar is specified which can be used to insert invariants into source code. Existential and universal quantifiers are supported.

H-SEND is optimised for WSNs in a variety of ways. For example, it minimises overhead by buffering messages it needs to send, and piggybacking them on the existing network traffic. Due to the hierarchical nature of the protocol, multinode-invariants can be checked efficiently at the closest parent node with all the required information.

\subsubsection*{Dynamic Invariant Detection $\cup$ Checking Engine (DIDUCE)}

DIDUCE \cite{diduce} is a tool which employs machine learning to dynamically generate hypotheses of invariants for a system at run-time; the invariants begin extremely strict, and are relaxed over time to allow for new correct behaviour. The machine learning aspect means that developers do not have to specify invariants themselves, which proves beneficial as accurately pinpointing the values necessary for fault-free operation is non-trivial. DIDUCE checks against the invariants continually during a program's operation and reports all violations detected at the end of the run, whereas Daicon merely presents the user with invariants found. For all its apparent usefulness, unfortunately DIDUCE was designed for large, complex systems rather than lightweight distributed systems with constrained resources (i.e. WSNs), so it is likely to prove infeasible to use this tool for GPD.

\subsubsection*{DAIKON: A tool for dynamically detecting invariants in sensor network applications}

One method of automatically inferring invariants in applications is the use of a prototype invariant detector, DAIKON \cite{daikon}, which uses execution traces to produce a list of likely invariants. The set of dynamically detected invariants depend on observed values and the invariants are an indication of the quality of a test suite. Daikon consists of two parts; a language specific front end and a language independent interference engine. The former executes the running program and accesses its runtime state to get the required information (consist of variables and their value). This information contains a subset of relevant variables and only these are written to the trace file. A variable is considered as relevant depending on the type of invariants targeted and whether they are accessible at an instrumentation point. This forms an input to the second part of the system - the invariant inference engine - which uses machine learning techniques and produces a set of detected invariants for the program.

The main challenge with these techniques is deducing the relevance of the invariants as it depends on the programmer's expereience and knowledge of the underlying system. However, these can be improved with techniques like exploiting unused polymorphism and suppressing invariants that are logically implied by other invariants. Daikon does not require the programmer to specify invariants for the application, however, including DIDUCE, it is not designed for distributed or resource-constrained systems like WSN.


\begin{comment}
\subsection{Quality of Service}
The area of Quality of Service (QoS) within Wireless Sensor Networks (WSN) is largely unexplored, due to the large differences between WSNs and traditional wireless networks. Traditional networks determine QoS based on high bandwidth allowance, as a result of high multimedia demands of applications. WSNs typically do not need to transfer this amount of data, and have a much lower bandwidth because of this. WSNs also have a wide range of different applications, and as a result, it is not clear how to develop transferable approaches to QoS  \cite{Akyildiz2002393}. 

QoS can be reduced to 'a set of service requirements to be met when transporting a packet stream from the source to its destination' \cite{Crawley98aframework}. With traditional networks, redundancy is often introduced to allow for high load/traffic, however redundancy in WSNs can often mean wasted energy usage which is often the main QoS measure in many protocols \cite{AkkayaYounis2003}.

Akyildiz et. al. \cite{Akyildiz2002393} suggested that QoS could be measured in two ways Application and Network. The Application defines measures such as coverage, number of active sensors and exposure, while the Network is concerned with delivering the QoS constrained data, while maintaining network efficiency (minimising resources).

Akyildiz et. al. further went on to describe the challenges specific to WSN; 
\begin{itemize}  
			\item Resource Constraints - Battery life, memory, bandwidth etc
			\item Unbalanced traffic - Traffic flows from large set of sensors into a small set of sink nodes
			\item Data redundancy - The re-transmission of data could result in wasted energy usage
			\item Network Dynamics - Failing nodes/wireless links, energy conservation, mobility etc
			\item Scalability 
			\item Multiple sink nodes - Each node could have a different set of requirements
			\item Packet Critically - Some data may need to flow through the network quicker than other pieces
			\item Multiple traffic types - Different pieces of data flowing through the network at the same time
\end{itemize}

QoS is a difficult term to define, mainly due to its various meanings and perspectives, because of this, measurements of quality must be generated based on the application involved, and the specific requirements of that application.


\subsection{MAC Protocols}

\subsubsection*{Energy-efficient MAC Protocol Designed for WSN for IoT (The submarine paper)}

\cite{6128220} discusses the energy efficiency of existing protocols, including the original adaptation of MAC for WSNs - Sensor MAC (SMAC). Describes the operation of SMAC, which uses a fixed listen/sleep cycle to reduce idle listening time and thus save energy. Goes on to mention two improvements on SMAC: first Timeout MAC (TMAC), in which the listen/sleep cycle is adapted according to network traffic, by means of a simple timeout mechanism; then $\mu$-MAC, which alters between contention and contention-free periods. The former is used to establish network topology and initialise sub-channels (collections of time slots), which are used in the contention-free period to transmit without collisions.

The authors then propose a power-controlled MAC protocol (PC-MAC), which determines the required transmission power level for a packet, thus aiming to save unnecessary energy usage when sending over short distances. This calculation assumes the physical layer of the nodes can transmit frames at one of a discrete set of power levels notified by the MAC layer. Once calculated, the minimum required power for each of a node's neighbours is stored in that node's Schedule and Power Level Table, an extension of the Schedule Table used for synchronising sleep cycles in SMAC. The protocol preserves the collision- and overhearing-avoidance properties of SMAC.
Authors report energy savings of between 50\% and 96\% for average node distances ranging from 10m down to 1m. These findings were generated solely using simulations, but assuming they hold for a hardware WSN the benefits for energy efficiency are significant enough to warrant serious consideration.


\subsubsection*{Energy Analysis of Four WSN MAC Protocols}

Four power-aware protocols based on the MAC framework implemented in TinyOS on TelosB motes and were tested using broadcast, convergecast and local gossip traffic patterns \cite{5751321}. Motivation: testing of the protocols side-by-side under controlled parameters, to rule out the innumerable extraneous factors that make direct comparison of separately-published protocols difficult.

Outlines history of power-saving MAC strategies, beginning with the duty cycles as described above in the operation of SMAC. Then describes the next development --- low-power listening (LPL), using transmission preambles and channel polling to reduce idle listening times. More advanced protocols use a hybrid of these two techniques.

Protocols tested:
\begin{enumerate}
	\item Scheduled Channel Polling MAC (SCP-MAC):
	\begin{itemize}
		\item Modification of LPL by waking up all neighbouring motes to listen at the same time. This leads to shorter preambles and duty cycles than typical LPL protocols (i.e. BMAC). However, all neighbours share a listening slot, so overhearing is common.
	\end{itemize}
	
	\item Asynchronous Schedules MAC (AS-MAC):
	\begin{itemize}
		\item Eliminates overhearing by assigning unique time slots for each mote to listen; the times when each mote wakes up to transmit/receive are determined by its internal Neighbour Table. At each wakeup, a mote polls for packet receptions, and motes transmit during a contention window overlapping with this wakeup slot. Loss of contention signals a retry during the recipient's next wakeup. AS-MAC uses sync packets and non-uniform offsets to offer unique receiver receptions slots, even in dense neighbourhoods.
	\end{itemize}
	
	\item Crankshaft:
	\begin{itemize}
		\item Similar to AS-MAC in that time is divided into frames, which are sub-divided into receiver slots. Frames include broadcast and unicast slots such that all neighbouring motes wake up for the all of the former, and only their own unicast slot. The ratio between the two types of slot is configurable at compile time, and the number of unicast slots is independent of the number of motes. As such, dense WSNs often feature multiple receivers contending for receptions during the same slot; clock synchronization in Crankshaft relies upon upper layers.
	\end{itemize}
	
	\item Broadcastable AS-MAC (BAS-MAC):
	\begin{itemize}
		\item During implementation of broadcasting in AS-MAC (using multiple unicast transmissions), it was noticed that a broadcasting mote must stay awake for the duration of every receive slot in the network; for nontrivial network sizes, this would be infeasible. As such, a separate protocol was created. BAS-MAC is based heavily on AS-MAC, but defines a broadcast interval; a time slot during which all neighbouring motes wake up simultaneously.
	\end{itemize}
\end{enumerate}

Measurement of the energy usage of each protocol was approached by recording the amount of time motes spent in each radio state, and multiplying each of these times by a constant representing the energy usage of that state per unit time. These constants were determined using an oscilloscope connected to a mote running AS-MAC (as energy consumption per state is protocol-independent). Minor alterations were made to each protocol to  ``level the playing field" in the case of slow/complex protocol initialisations and inapplicable network assumptions.

The results of the experiments, organised by network traffic type, are as follows.
\begin{enumerate}
	\item Local gossip:
	
	AS-MAC demonstrated highest energy efficiency for this traffic type, with Crankshaft and BAS-MAC both using approximately 40\% more energy (due to increased idleness caused by the lack of broadcasts), and SCP-MAC using more energy still due to its overhearing avoidance being inapplicable due to the packet-based TelosB motes.
	
	\item Convergecast:
	
	AS-MAC shows the best overall performance again, though for receiving motes the energy usage of SCP-MAC is a very close second. Again, the unused second wakeup for Crankshaft and BAS-MAC leads to idling.
	
	\item Broadcast:
	
	As AS-MAC is inherently poorly suited to broadcasts, its sender used almost triple the energy of that of the next-least efficient protocol, though its receivers were most efficient by a small margin. Thus, in a single-hop network where the base station node is not battery powered, it is a good choice. However, SCP-MAC performed best overall, by a significant margin over both Crankshaft and BAS-MAC.
\end{enumerate}

The authors conclude that no single protocol excels in all circumstances; AS-MAC and SCP-MAC are more efficient with non-broadcast and broadcast traffic, respectively, where BAS-MAC performs moderately well in each scenario. However, as the base station for our network will typically be a laptop or desktop, the disadvantage of AS-MAC in broadcasting could potentially be ignored. The authors also suggest using a framework such as MLA to host a suite of protocols suited to different tasks, each of which may be swapped in to match the current circumstances.

This last point may well prove to be beyond the scope of our project, however the results for individual protocols may prove useful. It is unfortunate that a fair comparison between these protocols and PC-MAC could not be made.

\subsubsection*{A Traffic Queue-aware MAC Protocol for WSNs}

This paper \cite{4469515} introduces the traffic Queue-aware Sensor MAC protocol (QSMAC), based on SMAC to predict amount of data traffic in a network.

After describing the operation of SMAC (as above), the authors highlights the problem of a fixed cycle duration in networks where traffic loads can fluctuate; an overflowing node buffer queue may cause packets to be discarded, causing additional energy expenditure in resending lost packets and debasing the network QoS.

The basic operation of QSMAC involves examining the average increase rate of data packets in a node's buffer queue. When this rate is more than one per second, packets are arriving faster than they can be processed using the default SMAC cycle, so the cycle duration is halved to double its effective processing speed. When the buffer is almost empty, the default cycle is restored.

Simulations show performance greater than that of SMAC in terms of packets' delay, energy consumption, packet reception ratio and network throughput. As above, if these benefits translate to real hardware, QSMAC potentially offers a real improvement over existing wireless MAC protocols. However, its use in our network will have to be carefully considered, as it is currently unknown whether fluctuating traffic loads will be a likely scenario.
\end{comment}

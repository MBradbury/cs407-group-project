% !TeX root = Report.tex
\section{Project Management}

\subsection{Methodology}
% Just notes...

\subsubsection*{Term 1}

From the beginning, our main objective was to develop a wireless sensor network (WSN) to investigate the debugging of distributed programs in such environments. However, WSN is a fast growing research area and encompasses a wide range of applications and practical aspects of sensor deployment. From our research, such applications include wildlife monitoring such as animal habitat and behaviour and environment monitoring like air pollution and volcanic activity. These applications where developed and deployed in real situations but for our project we decided to take on a more general approach by investigating the issues regarding debugging and fault detection in wireless sensor networks.

As a result, we spent the first few weeks researching on the different aspects of wireless sensor networks as well as the appropriate tools and platform to develop our system. These include;

\begin{enumerate}
	\item Operating systems and simulation platforms
		\begin{enumerate}
			\item TinyOS with TOSSIM
			\item Contiki with Cooja
		\end{enumerate}
	\item Routing protocols and sensor communication
	\item Predicate evaluation
	\item Monitoring and interface tools
	\item Logging and error reporting
	\item Sensor limitations (e.g. energy and power usage)
\end{enumerate} 

%Research
We managed to gather a lot of information on these topics and had to pick the most suitable ones to implement in our system. However, some concepts such as broadcasting consists of numerous methods where each are tailored to work best in different systems and working environments. Since this is beyond our practical knowledge, we had to implement most of the methods and tested it on the system to determine the most efficient one. With this proceeding, it was hard to define the overall scope of our project since there were many factors that had to be considered. Nevertheless, we produced a list of all the possible algorithms and protocols to implement and ranked them according to its relevance. We followed a 'dive-in' procedure by implementing the most relevant components first and updated the list by removing or adding algorithms. To facilitate this, we employed a prototyping development model in the hopes of obtaining a more defined project scope by term 1.

The requirements of the project was also not fully understood at the time and we were unsure of the efficiency of our algorithms so we decided to go for a prototyping approach as part of an evolutionary process model. This allows us to start with a set of core product and system requirements (i.e. the operating system) and then incorporate other extensions as the requirements are better understood. We also had frequent meetings with our supervisor to discuss the best ways to implement our system and any additional components. Where a new idea is defined, a prototype is modelled and constructed and is re-evaluated by our supervisor, who provides feedback which is used to further refine the requirements. This process is repeated until the supervisor is satisfied with the function of the system, while at the same time enabling us to better understand what needs to be done.

By the end of the first term, the scope of the project had reached a more defined level as we had decided what is to be implemented and what has been disregarded from the provisional list of algorithms. This was demonstrated in our first delivery - the poster presentation, where we had to chance to explain our project at a high level to our supervisors and other professors.
%Can state here what was said in the feedback form.

\subsubsection*{Term 2}

At this stage, most of the algorithms and protocols were implemented independently and treated as individual components to the core system. We decided to do it this way because of the benefits of modularity where the principles of "Separation of Concerns" allows for a more manageable set of modules. The composition of these components such as predicate checking and neighbour data retrieval was done during the entire winter vacation and the beginning of the second term.
%We can mention here a small list of the algorithms that needed to be integrated together

In the second term, we continued following the evolutionary process model but at a more relaxed approach which focuses on finishing the system components and preparing the gathering of results. Since the project scope was clear and we had a more directed goal,less meetings with the supervisor was needed, but we still required him to evaluate our work and provide feedback. Ensuring that the development and debugging phase to complete in time required the effective scheduling of tasks and prioritising work force to the critical ones. By the third quarter of the term, progress had fallen to a halt due the high number of coursework deadlines and this meant that team members where preoccupied with other duties. To address the issue, a meeting was held prior to the halt where we discussed our recovery plan and actions to take once members was free. This allowed team members to continue working where they left off and avoid any delays.
% Needs to be re-written 

%Notes
%In the first half of the second term, most of the work was focused towards completing the implementation of sensor communication, converting existing codes into libraries and the development of network visualisation. 

%By the second half, many people where preoccupied with other module coursework and progress was slowed down. Towards the end of the term, some of the algorithms such as Tree Aggregation needed optimization 

	%\item The main task was to integrate the tasks that worked together and convert them into libraries which would be needed at a high levels of the system
	%\item Visualisation and Base station
	%\item Optimisation
	%\item Debug and testing
	%\item Run simulation and gather results
\subsection{Role Allocation and stakeholders}

We decided to allocate roles in the second week after we had the first week to perform research into the problem and find out what has been done. The following were how we assigned roles, although we intend for these to be flexible:

\begin{table}[H]
\centering
	\begin{tabular}{| c | c |}
		\hline
		Name & Role\\
		\hline
		Matthew Bradbury & Group Leader\\
		Daniel Robertson & Project Manager\\
		Amit Shah & Technical Leader\\
		Ivan Leong & Developer and Tester\\
		Joe Yarnall & Developer and Tester\\
		Tim Law & Developer and Researcher\\
		\hline
	\end{tabular}
\end{table}

Whilst the entire team was responsible for system design and development, it was critical to have a group leader and other leading roles to facilitate and provide resources to the team. Their individual responsibilities is as follows:

\begin{itemize}
	\item[] {\bf Matthew Bradbury} - Group Leader
	
	Responsible for defining and managing the entire project.
	Responsible for allocating tasks to its team members and providing direction for meeting the project objectives.
	Understands all the workings of the project and communicating the ideas to the stakeholders (i.e supervisor).
	Ensures that everyone is performing their roles accordingly and resolving conflicts within the team.
	Monitors the progress of the team and providing motivation when the project is slowing down.	

	\item[] {\bf Daniel Robertson} - Project Manager
	
	He supports the group leader by facilitating the management of the project and ensuring that the overall work is done on schedule. He arranges weekly meetings and provides a summary report prior to each meeting and records all discussions.
	
	\item[] {\bf Amit Shah} - Technical Leader
	
	Responsible for overseeing the work done by other members and ensures that all code written by the team meets the technical specification and design requirements of the project.	
	
\end{itemize}

Need to mention our supervisors role here: What does he want us to accomplish? His interests?

\subsection{Schedule}

\begin{table}[H]
	\centering
	\begin{tabular}{| l | l | l | l | l | l | l |}
	\hline
	Task Description & \multicolumn{6}{l|}{Time Allocated (Weeks)}\\
	~ & Amit & Dan & Ivan & Joe & Matt & Tim \\
	\hline
	\hline
	\multicolumn{7}{|l|}{\textbf{Term 1} - Developing for application predicate checking} \\
	\hline


	Research around the Problem & 2 & 2 & 2 & 2 & 2 & 2\\
	Writing Specification & 1 & 1 & 1 & 1 & 1 & 1\\
	H-SEND Implementation & 3.5 & ~ & ~ & 3.5 & ~ & ~\\
	``Send to Base'' Implementation & ~ & ~ & ~ & ~ & 2 & 1.5\\
	Clustering Implementation & ~ & 3.5 & 2 & ~ & ~ & ~\\
	Aggregation Tree Implementation & ~ & ~ & ~ & ~ & 1.5 & ~\\
	Develop Visualisation Tool & ~ & ~ & 1.5 & ~ & ~ & 2\\
	Develop Predicate Language Runtime & ~ & ~ & ~ & ~ & ~ & 2\\
	Testing and Adapting to Physical Nodes & 2 & 2 & 2 & 2 & 2 & 2\\
	Message Logging & ~ & ~ & ~ & 1 & ~ & ~\\
	Poster Creation and Presentation preparation & 1.5 & 1.5 & 1.5 & 1.5 & 1.5 & 1.5\\

	\hline
	\hline
	\multicolumn{7}{|l|}{\textbf{Term 2} - Developing for network predicate checking and where the predicate is checked} \\
	\hline
	
	Additional Research & 1 & 1 & 1 & 1 & 1 & 1\\
	Improving Dynamic Predicate Specification & 2 & 2 & ~ & ~ & 2 & 2\\
	Develop Visualisation Tool & 2 & ~ & 2 & 2 & ~ & 2\\
	Modify Algorithms to Selectively Evaluate Predicates & ~ & 2 & 2 & 2 & 2 & ~\\
	Performance Testing & 1.5 & 1.5 & 1.5 & 1.5 & 1.5 & 1.5\\
	Testing and Adapting to Physical Nodes & 1.5 & 1.5 & 1.5 & 1.5 & 1.5 & 1.5\\
	Report Writing & 2 & 2 & 2 & 2 & 2 & 2\\
	
	\hline
	
	\end{tabular}
\end{table}

\subsection{Delivery plan}

\subsubsection{Gantt Chart}

Gantt Chart comes here

\subsubsection{Deliverables}

\begin{enumerate}
	\item Specification
    \item Progress Poster Presentation
    \item Final Group Report
    \item Individual Report
    \item Final Project Presentation
\end{enumerate}

\subsection{Risk Analysis}
%Potentially the issues below can form part of risk

\begin{center}
	\begin{longtable}{| p{5cm} | l | p{8cm} |}
	\hline
	Risk & Severity & Mitigation\\
	\hline	
	
	Hardware and Software failure
	\begin{itemize}
		\item Report and research
		\item Software code
	\end{itemize}
	 & 8 & To prevent the risk of data lost, we used a Bitbucket repository to save and store all our documents and program coding online. This also allows the concurrent development of our project and ensuring that all team members are up to date.
	\\ \hline
	
	Team member illness and absence
	& 4 & There would be circumstances where team members would be unavailable to work on the project for a short period of time. To ensure that work is progressive, necessary measures were taken. For example an absent member can temporary delegate his current task to a different member.
	\\ \hline
		
	Meeting deadlines and schedules
	& 8 & It was important that every milestone of the project was met so that no marks were penalised. The progress of the project was monitored and we had weekly meetings to ensure that everyone was on track of their given tasks.
	\\ \hline
	
	Integration of code based on simulated data to real hardware
	& 5 & Most of the code that was developed in the beginning was tested with the cooja simulator. It would be unlikely to obtain similar results when deployed on the actual hardware due to a wide range of environmental factors such as //NEED EXAMPLE HERE.
		\\ \hline
		
	Defining the scope of our project
	& 5 & There are many issues and research problems that revolve around the topic of wireless sensor networks.

It is important to define the scope of the project to prevent scope creep and necessary expansion of the project.

Need to keep track of all new ideas proposed by the supervisor and raise any change requests to the project.

		\\ \hline
		
	Safety of the sensor nodes
	& 6 & Each of the sensor nodes cost xxxGBP. The sensors must be handled with care to prevent any cost of damage. //WHAT IF WE DO??
		\\ \hline
		
	Components of the project that could not be implemented with current tools.
	& 6 & Research on alternative methods
	\\ \hline		
		
	\end{longtable}

\end{center}

\subsection{Resources}
% Can explain more on the software we used to share resources, communicate and discuss ideas.

\subsubsection{Working Concurrently}

We signed up for a Git repository on BitBucket \cite{bitbucket} where we plan to commit all the work we produce. We initially had an issue that free private repositories hosted on BitBucket have a maximum of 5 participants, whereas we had 6 group members. Fortunately when new users sign up to the services from an invite, the person that sends the invite gets additional capacity. This meant that the person who created the repository ended up with enough capacity for all members to access the account.

\subsubsection{Communication}

Facebook

\subsubsection{Development Platform}

Contiki, Simulator (Cooja), DCS machines, etc 

\subsection{Work Overview}

This section highlights the different tasks that was undertaken by each group member for every week of term 1 and term 2. At every weekly meetings, the project leader would discuss the progress of the tasks that was assigned in the previous week and any new tasks for the current week would be recorded as shown in the table below.
%perhaps number the table below 

\subsubsection*{Term 1}

\begin{center}
	\begin{longtable}{| l | p{9cm} | p{3.5cm} |}
	\hline
	Week & Activities & Task Allocation\\
	\hline
	1 & \begin{enumerate}
			\item Meet up with Supervisor and discuss project direction
			\item Research on related work and what we can develop, before settling on main aims
			\item Investigate two OS and their corresponding simulators: TinyOS with TOSSIM and Contiki with Cooja
		\end{enumerate} &
	\begin{enumerate}
		\item[] Amit: All
		\item[] Dan: All
		\item[] Ivan: All
		\item[] Joe: All
		\item[] Matt: All
		\item[] Tim: All
	\end{enumerate}
	\\ \hline

	2 & \begin{enumerate}
			\item Research if Cooja can be extended through the use of plug-ins (with the aim of extending it to replay traffic logs)
			\item Research Clustering algorithms and find implementations
			\item Develop a temperature dissemination application to learn Contiki and Cooja
			\item Research TinyOS and TinyDB and see if they could be applied to predicate checking
			\item Investigate performance of different MAC protocols
			\item Investigate the feasibility of live monitoring of the network
			\item Investigate QoS and how it may be applied to real life WSN deployments
			\item Produce a literature review on chosen topic
		\end{enumerate} &
	\begin{enumerate}
		\item[] Amit: 1, 7, 8
		\item[] Dan: 5, 8
		\item[] Ivan: 6, 8
		\item[] Joe: 4, 8
		\item[] Matt: 3, 8
		\item[] Tim: 2, 8
	\end{enumerate}
	\\ \hline
	
	3 & \begin{enumerate}
			\item Research DICAS
			\item Research Send to Base
			\item Research Daicon
			\item Research DIDUCE
			\item Research H-SEND
			\item Research Sympathy
			\item Write Specification
		\end{enumerate} &
	\begin{enumerate}
		\item[] Amit: 2, 7
		\item[] Dan: 4, 7
		\item[] Ivan: 3, 7
		\item[] Joe: 1, 7
		\item[] Matt: 6, 7
		\item[] Tim: 5, 7
	\end{enumerate}
	\\ \hline
	
	
	4 & \begin{enumerate}
			\item Finish Specification
			\item Work out how to get MSP430-GCC and the Cooja working on Joshua
			\item Begin developing aggregation tree example
		\end{enumerate} &
	\begin{enumerate}
		\item[] Amit: 1, 2
		\item[] Dan: 1
		\item[] Ivan: 1
		\item[] Joe: 1
		\item[] Matt: 1, 3
		\item[] Tim: 1
	\end{enumerate}
	\\ \hline
	
	
	5 & \begin{enumerate}
			\item Get used to developing in C and learn how to use the Contiki libraries
			\item Finish developing aggregation tree
			\item Start implementing clustering algorithms
			\item Start implementing H-SEND
		\end{enumerate} &
	\begin{enumerate}
		\item[] Amit: 1, 4
		\item[] Dan: 1, 3
		\item[] Ivan: 1, 3
		\item[] Joe: 1, 4
		\item[] Matt: 1, 2
		\item[] Tim: 1
	\end{enumerate}
	\\ \hline
	
	
	6 & \begin{enumerate}
			\item Start Implementing GUI
			\item Implementing clustering algorithms (Regular and Hierarchical)
			\item Implement H-SEND
			\item Implement a local predicate checker and reporter
		\end{enumerate} &
	\begin{enumerate}
		\item[] Amit: 3
		\item[] Dan: 2
		\item[] Ivan: 2
		\item[] Joe: 3
		\item[] Matt: 4, 2
		\item[] Tim: 1
	\end{enumerate}
	\\ \hline
	
	7 & \begin{enumerate}
			\item Library-ify Cluster implementations
			\item Implement LEACH  using existing implementations as a base
			\item Work out how to interface the GUI and the base station
			\item Continue developing GUI
			\item Get in contact with Sain Saginbekov and sort out when we can use sensor nodes
			\item Contact Victor Sanchez and inform him when we are using sensor nodes
			\item Implement multi-hop predicate checking
			\item Slim down eLua enough for us to run it on the sensor nodes, to allow for dynamic predicate specification
			\item Research and start implementing network traffic logging and reporting
		\end{enumerate} &
	\begin{enumerate}
		\item[] Amit: 7
		\item[] Dan: 1, 2
		\item[] Ivan: 9
		\item[] Joe: 5, 6, 7
		\item[] Matt: 8
		\item[] Tim: 3, 4
	\end{enumerate}
	\\ \hline
	
	8 & \begin{enumerate}
			\item Start work on poster with the aim of finishing it by the end of week 9
			\item Detect and report neighbours for display in the visualisation tool, aim to be able to work in mobile networks, with gained and lost neighbours
			\item Log network data and report it to the network (for visualisation tool to replay traffic)
			\item Finish predicate language parser and code generation
			\item Tidy up n-hop HSEND
			\item Integrate n-hop HSEND with predicate language virtual machine
			\item Experiment with interference affecting node communications
		\end{enumerate} &
	\begin{enumerate}
		\item[] Amit: 1, 5, 6
		\item[] Dan: 1, 3, (~7)
		\item[] Ivan: 1, 2, (~7)
		\item[] Joe: 1, 2, (~7)
		\item[] Matt: 1, 4, 6
		\item[] Tim: 1, 4
	\end{enumerate}
	\\ \hline

	9 & \begin{enumerate}
			\item Create poster for presentation
		\end{enumerate} &
	\begin{enumerate}
		\item[] Amit: 1
		\item[] Dan: 1
		\item[] Ivan: 1
		\item[] Joe: 1
		\item[] Matt: 1
		\item[] Tim: 1
	\end{enumerate}
	\\ \hline

	10+ & \begin{enumerate}
			\item Code generation for predicate language
			\item Network visualisation (node neighbours)
			\item GUI to Node interface
			\item Message logging
			\item Neighbour Reporting
			\item Turn existing code into libraries
			\item Make existing code handle generic data
			\item Integrate predicate evaluating VM and neighbour data retrieval
			\item Write up notes on work done this term
		\end{enumerate} &
	\begin{enumerate}
		\item[] Amit: 6, 7, 8, 9
		\item[] Dan: 4, 5, 9
		\item[] Ivan: 3, 5, 9
		\item[] Joe: 6, 7, 8, 9
		\item[] Matt: 6, 7, 8, 9
		\item[] Tim: 1, 2, 9
	\end{enumerate}
	\\ \hline
	
	\end{longtable}
\end{center}

\subsubsection*{Term 2}

\begin{center}
	\begin{longtable}{| l | p{7.5cm} | p{5cm} |}
	\hline
	Week & Activities & Task Allocation\\
	\hline
	1 - 2 & \begin{enumerate}
		\item Continue network visualisation
		\item Create a test suite for network visualisation
		\item Interface Base station node with desktop application
		\item Develop neighbour detection and reporting to sink
		\item Integrate predicate evaluating, predicate VM and neighbour data retrieval
		\item Finish parser and compiler of predicate language
		\end{enumerate} &
	\begin{enumerate}
		\item[] Amit: 5
		\item[] Dan: 4
		\item[] Ivan: 3
		\item[] Joe: 5
		\item[] Matt: 5, 6
		\item[] Tim: 1, 2
	\end{enumerate}
	\\ \hline

	3 & \begin{enumerate}
		\item Continue network visualisation
		\item Create a test suite for network visualisation
		\item Interface Base station node with desktop application
		\item Develop type checker for predicate language
		\item Find bug that is causing unicast communications to fail
		\item Develop a way to handle packets larger than the maximum broadcast size
		\end{enumerate} &
	\begin{enumerate}
		\item[] Amit: 5
		\item[] Dan: 6
		\item[] Ivan: 3
		\item[] Joe: 5
		\item[] Matt: 4
		\item[] Tim: 1, 2
	\end{enumerate}
	\\ \hline

	4 & \begin{enumerate}
		\item Continue network visualisation
		\item Create a test suite for network visualisation
		\item Interface Base station node with desktop application
		\item Find bug that is causing unicast communications to fail
		\item Develop a way to handle packets larger than the maximum broadcast size
		\item Develop some containers for use in our applications (lists and maps)
		\end{enumerate} &
	\begin{enumerate}
		\item[] Amit: 4
		\item[] Dan: 5
		\item[] Ivan: 3
		\item[] Joe: 4
		\item[] Matt: 4, 6
		\item[] Tim: 1, 2
	\end{enumerate}
	\\ \hline

	5 & \begin{enumerate}
		\item Continue network visualisation
		\item Create a test suite for network visualisation
		\item Interface Base station node with desktop application
		\item Improve reliability of application on physical hardware
		\item Refactor N-Hop-Req to split up request and reply phases
		\item Develop a way to handle packets larger than the maximum broadcast size
		\item Optimise and debug Tree Aggregation and Predicate Checking
		\end{enumerate} &
	\begin{enumerate}
		\item[] Amit: 4, 7
		\item[] Dan: 6
		\item[] Ivan: 3
		\item[] Joe: 5
		\item[] Matt: 7
		\item[] Tim: 1, 2
	\end{enumerate}
	\\ \hline

	6 - 10 & \begin{enumerate}
		\item Implement nhopflood
		\item Implement Event based data sending
		\item Implement Event-based predicate evaluation (PELE, PEGE)
		\item Implement Global predicate evaluation (PEGP, PEGE)
		\item Implement GUI
		\item Get bidirectional communications between motes and desktop working
		\item Develop a way to handle packets larger than the maximum broadcast size (multipacket)
		\item Debug and Test
		\item Implement failure responses for predicate evaluation
		\item Run simulations to gather results
		\end{enumerate} &
	\begin{enumerate}
		\item[] Amit: 3, 4, 8, 9, 10
		\item[] Dan: 1, 7, 8
		\item[] Ivan: 6, 8
		\item[] Joe: 8, 9
		\item[] Matt: 1, 2, 3, 4, 6, 7, 9, 10
		\item[] Tim: 5, 8
	\end{enumerate}
	\\ \hline
	
	\end{longtable}
\end{center}



\subsection{Management Issues}

\subsubsection{Group Members Without Internet}
Unfortunately two of our group members were without internet for the first 3 weeks of term. This was an issue because they were unable to just work in the DCS labs because the computers there didn't have the software required (such as VirtualBox or the WSN simulators). To work around this, those two members were given tasks that could be accomplished with their own machines and without internet. Once they obtained internet they were given tasks, that access allowed them to accomplish.

\subsubsection{Society Exec Clashes}
Two members of our team are on the exec of Warwick Societies, this means that at certain parts of the year they were in high demand for those jobs. For instance during the first few weeks of term during Freshers' Fortnight, the exec members were required to be involved in many of their events that they were running. However, after these first weeks the demand on their time decreased and made balancing the time been the demands of this project and their exec roles much easier.

\subsubsection{High number of coursework deadline in term 2}
In the second term, there where many coursework deadlines for other course modules and some of the team members where unable to contribute towards the project for some periods of time. On several occasion, the team leader had to put pressure on us and ensure that some time was delegated to project work. 

\subsection{Legal, social, ethical and professional issues}


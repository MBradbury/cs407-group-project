% !TeX root = Report.tex
\section{Model}

As stated in the introduction we are investigating developing tools to aid in debugging the distributed programs running on wireless sensor networks. We aim to do this by implementing libraries that use different technique to check predicates, with a focus on correctly evaluating these predicates. In order to do this we first need to model the situation that our libraries will be used in to aid in the explanation of our findings.

To begin with we model a wireless sensor network as a graph $G = (V, E)$ where $V$ is a set of nodes. If we have two nodes $u$ and $v$ that can communicate $\{u, v\} \in E$, we assume that this communication is bi-directional. Every node in the network has a unique identifier 

\begin{mydef}
\emph{$m$-Hop Neighbourhood}: Given a node $n$ the $m$-hop neighbourhood of that node is the set of nodes that are within $m$ hops of node $n$. When we refer to this neighbourhood it will not contain the node $n$.
\end{mydef}

As we are dealing with simulators and physical hardware, there is no assumption of reliable links between nodes. Techniques will be described later on that increase the probability of a message reaching its target, but do not ensure it.

Every node in the network is assumed to have the same hardware and thus the same capabilities. As every node has the same capabilities, every node has similar properties such as: transmission range and initial power levels. We assume that the network is working in isolation to other networks and is not receiving other electromagnetic interference.

When checking predicates we much deal with the notion of the \emph{correctness of evaluation} of a predicate. As can be seen from the definition this involved the notion of time because the values used to evaluate the predicate can change over time.

\begin{mydef}
\emph{Correctly Evaluated Predicate}: A predicate $P$ is correctly evaluated at time $\tau$ if the results of evaluating that predicate with global knowledge at time $\tau$ is the same as the result of predicate $P$.
\end{mydef}

To evaluate a predicate there is some function $PE$ that takes a mapping from node identifiers to a user-defined structure of valued about that node and returns a boolean. By executing this function the result of a predicate is obtained. The input can be obtain through various means. The next chapter will discuss how this function is evaluated and how the required information can be disseminated to the node that requires it to perform the evaluation.

\subsection{Algorithm Descriptions}



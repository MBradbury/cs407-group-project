% !TeX root = Report.tex
\section{Introduction}

\subsection{What is a Wireless Sensor Network?}

A wireless sensor network, or WSN for short, is a collection of networked sensors called Motes; these sensors are capable of short range wireless communication and they have the ability to sense their surrounding environment\cite{Mica2002,TankBible}. The network forms a distributed system that can perform a variety of distributed algorithms, usually data gathering and similar tasks. To communicate, each node is equipped with a radio that allows them to send and receive messages to neighbouring nodes within a limited range. To sense the environment motes typically have a range of embedded sensors such as heat, light, humidity and many others. They typically contain a simple central processing unit (CPU), which is programmed to control the hardware on the motes. The CPU also processes events which are triggered by the hardware (such as messages being sent and received) and it also handles any other computation necessary for the operation of the system. As the platform is designed to be mobile the motes do not operate on a mains power supply, instead they run off stored energy in a battery. WSNs is a field of Computer Science that is currently the focus of much research and wireless sensor networks have a wide range of practical applications that stretch from battlefield intelligence for the military\cite{Akyildiz2002393,1368897,1457970} to industrial process monitoring for manufacturing companies\cite{?}.

A defining characteristic of designing applications for wireless sensor networks is the restricted and finite energy supply available to each node. Therefore, wireless sensor nodes tend not to use expensive broadcasting protocols such as IEEE 802.11 \cite{Mica2002}, but instead use much simpler alternatives to save energy. For example wireless protocols such as IEEE 802.15.4 ZigBee \cite{1253873, 4014617} are designed to be used by wireless sensor networks and have a lower energy usage associated with them. Some applications rely on even lower level behaviour specified by a certain MAC layer \cite{5751321,4469515,Polastre:2004:VLP:1031495.1031508,1019408,Buettner:2006:XSP:1182807.1182838}, these applications involve a trade-off between development time and energy usage. Where simplicity is often sacrificed for decreased energy usage. Using these simple protocols unfortunately has the downside of meaning that broadcasts are subject to several types of collisions and message losses. So it is very important that the software running on the nodes is designed to handle these cases. Being battery powered means that development of applications for Wireless Sensor Networks is fundamentally limited to maximising the system's lifetime so that the highest utility can be achieved from the network.

As wireless sensor nodes operate in harsh outdoors condition\cite{SzewczykPMC04, Werner-Allen:2006:FYV:1298455.1298491}, there is a high probability of them failing. These faults can range from hardware damage caused by environmental conditions or tampering, software bugs, or simply a denial of service caused by nodes running out of power. So algorithms and software need to be designed to handle these potential failures, otherwise they risk catastrophic failure when they encounter these issues.

Wireless sensor nodes are designed with the intention for them to operate in remote and traditionally unreachable locations with no human input for the lifetime of their operation\cite{1437066}. Given this a defining characteristic of a WSN system is that any applications developed for it must be self-configuring in nature; this is, the system must be able to organize itself and the network with no external input\cite{1368897}. Evidently, solutions to this issue are often considered hand in hand with the problem of network robustness and fault tolerance.   

While a limited energy source, self-configuration and network robustness are the predominant characteristic of a wireless sensor node, there are numerous other traits or issues that can be considered. For instance it is possible for these nodes to be mobile (for example an ad-hoc network of PDAs or motes built into soldiers helmets) \cite{4224091}, this can lead to very interesting behaviour in handling communication between these nodes.

%TODO: MORE OBSCURE WSN BEHAVIOUR CONSIDERATIONS

\subsection{The Problem - Debugging Distributed Systems}

Developing a distributed system is considered a particularly challenging task, more so than a traditional application, there are several reasons for this. Firstly, Within a distributed system multiple processes must execute in parallel, this means that variables may be updated independently or in repose to other processes which can lead to a myriad of synchronisation and timing issues that the developer must account for. Secondly, traditional programming languages are not well suited to develop distributed programs\cite{93692,345131}.

In any system, software or otherwise, developed by humans there is the potential for mistakes. Mistakes can be benign or they can cause unintended behaviour and system failures. Developing tools to detect these \emph{bugs} and notify the developer so they can be corrected is an incredibly important part of any toolchain. For example the GNU toolchain has utilities such as gdb\cite{?}, this allows developers to place breakpoints in code which will halt the programs execution at that state so that it can be examined. There are also numerous other tools that look for memory issues (valgrind\cite{Bond:2007:TBA:1297105.1297057,Nethercote:2007:SBM:1254810.1254820,seward2004valgrind}), security flaws (TOOL NAME\cite{?}) and many other classes of bugs.

Developing distributed systems is a difficult task; however, debugging these systems can be even more challenging\cite{345131}. When considering a distributed system if you want to examine the state of a system at a given point you cannot simply set a breakpoint in your local binary. The solution to this debugging is non-trivial, this is due to the difficulties that arise from distributed systems being non-deterministic in nature due to message communications \cite{1676929,Joyce:1987:MDS:13677.22723,Fagerstrom:1988:DTD:55823.55833}. Be it when the message started transmission, how long it took, if it succeeded or in what order transmissions occurred. So every time a distributed program is run it is possible for a different result to be obtained, due to the different order of execution. This goes against one of the usual assumptions of debugging traditional applications where it is assumed that one execution with a set of inputs will execute in exactly the same way again with the same set of inputs \cite{?} (i.e. determinism).

As the execution may be different each time in a distributed system it is not suitable to wait for a bug to occur, and then try to work out where it is. Rather, the system needs to be self-evaluating its state as it executes the distributed program; if a fault is detected, then the debugging tools will report the issue. One way to do this it to test if the system satisfies some global predicate, of which there has been much work to find and check different classes of these predicates \cite{553309,345831,277788}. However, of all the work that has been done, little of it has focused on wireless sensor networks where an important focus is perhaps the trade-off between accuracy and the report-ability of a predicate with the aim of reducing energy usage. In this paper we will discuss our development of just such a set of tools, we intend to focus on developing a system that can accurately evaluate predicates and provide useful information about real sensor networks running outside of a simulator.

\subsection{Related Work}

\subsubsection{Fault-Error-Failure Cycle}
%TODO: Joe: Needs Work

It should be clear that the types of predicates are important to consider when developing a predicate checking mechanism. Also important are the types of errors that these predicate checking algorithms can detect. To understand this it is first important to understand how errors can arise, which can be done by examining the fault-error-failure cycle. This cycle says that a \emph{fault} once caused by either some external influence (e.g. radiation leading to bit-flips in memory \cite{1017791}) or internal influence (e.g. code bugs) will lead to an \emph{error}, this is the \emph{activation} step. An error is the manifestation of the fault (e.g. memory holding the incorrect value). An error then leads to a \emph{failure} in the step called \emph{propagation}, the failure of the system is an observable deviation from the system's specification (e.g. allowing doors to be opened that should remain closed). It is not always the case the faults lead to errors, or errors lead to failures, sometimes multiple faults or errors are respectively required to cause a single error or failure. \cite{1335465}

There is a choice of what should be measured and checked in predicates; should faults, errors or failures be measured? Faults cannot be measured \cite{?} in a way that is possible for a mote, so they are discounted. That leaves measuring errors and failures. Typically failures would be the event being measured \cite{?} as that is what arises after an error actually causes the issue to happen. However, errors can also be measured if there is a dedicated program checking the state and comparing it to an expected state \cite{?}. For example an ECC (error correcting code) such as a Hamming Code can be used to detect and correct an error (in this case a bit-flip) in some memory after a fault (such as a voltage surge ) \cite{hamming1950error}.

Much of what has been discussed has involved transient faults such as those caused by environmental conditions, however, there is a class of faults that are a lot more common and much easier to resolve - faults caused by software bugs. These faults can lead to programs ending up in the wrong state and performing incorrectly. There has been a certain amount of work that looks into detecting traditional distributed system bugs (such as deadlock \cite{5587352,5284172}) in wireless sensor networks. However, there has been little work in looking into providing tools to aid in system debugging.

\subsubsection{Classes of Distributed Predicates}
%TODO: Joe: Needs Work

To begin with it is important to understand what predicates are relevant to distributed systems. First off we have a distinction between global and local predicates, global predicates involve taking a consistent global snapshot of the system and checking whether the snapshot satisfies the global predicate \cite{277788} and local predicates instead work with a subset of the network \cite{553309}. These predicates have also have a notion of stability, a stable predicate will remain true once it has turned true (e.g. termination), whereas unstable predicates can alternate between true and false. Finally there is a distinction between weak and strong, where a weak predicate holds if there exists an observation where the predicate is true and a strong predicate holds if it is true for all observers of the distributed computation\cite{553309,Cooper:1991:CDG:127695.122774}. Knowing what classes of predicates there are is important because when checking certain properties of a system a certain class of predicate will be required and thus a certain implementation will be needed to ensure the predicate is correctly checked. An example of this is when running an algorithm using global snapshots to detect stable predicates, that same application may not be suitable to detect unstable predicates because the predicate could switch to false and then back to true before the next snapshot.

\subsubsection{Existing Sensor Network Predicate Checking Tools}

There are a number of existing predicate checking solutions that have already been developed that are more practical focused than the aforementioned theoretical work into global predicate detection.

\paragraph{H-SEND} One of these solutions is H-SEND \cite{herbert2007adaptive}, which stands for Hierachichal SEnsor Network Debugging. H-SEND is a framework for detecting faults in sensor networks, it was designed to minimise energy consumption and be capable of handling very large networks. As part of the implementation developer must specify invariants within their code using a custom grammar, these invariants are then semi--automatically inserted during compilation. If an invariant is violated at runtime actions are taken (such as increased logging frequency, or an error message to the base station), using these responses developer can use the information to fix the software, which possibly include uploading a patched version of the firmware.

Of the invariants that can be specified, there are typically three different dichotomies: (i) Local vs. Multi--node, (ii) Stateless vs. Stateful and (iii) Compile--time vs. Run--time. The first indicates whether the predicate needs information about the node it is being evaluated on (Local) or other nodes in the network (Multi). The second is if the invariant depends on the node's execution state (Stateful) or if it doesn't (Stateless). The third indicates if the invariant involves values that are fixed at compile time (such as integer constants) or if it compares against values obtained during run-time (such as neighbouring states or previous states).

H-SEND is optimised for WSNs in a variety of ways. For example, it minimises overhead by buffering messages it needs to send, and piggybacking them on the existing network traffic. Due to the hierarchical nature of the protocol, multinode-invariants can be checked efficiently at the closest parent node with all the required information.

\paragraph{Sympathy} One of the projects that is summarised by H-SEND's authors \citeauthor{herbert2007adaptive} in \cite{herbert2007adaptive} is a method for identifying and localizing failures called Sympathy \cite{ramanathan2005sympathy}. Sympathy is intended to be run either in pre- or post-deployment environments where it collects data from distributed nodes at a sink. When insufficient data is received it is taken to imply that there exists a problem (insufficient data is component defined). The idea is that by monitoring data (both actively and passively) between components the system can identify what kind of failure occurred in a certain area of the network. Both of which are very useful when trying to debug a failure.

It does, however, have some downsides. The first is that there is assumed to be no traffic and thus no application traffic or network congestion. These are real issues especially when applying this kind of debugging to a high throughput sensor network. There are also a number of spurious failure notification, which the authors are working on reducing, by applying a Bayes engine.

\paragraph{DAIKON} Following Sympathy's attempts to implement a Bayes engine to allow learning to better help classify response messages, there has been work on being able to automatically detect invariants in a system. DAIKON \cite{daikon} is a system that uses execution traces to produce a list of likely invariants by way of automatically inferring the invariants through the use of a prototype invariant detector.

The set of dynamically detected invariants depend on observed values and the invariants are an indication of the quality of a test suite. Daikon consists of two parts; a language specific front end and a language independent interference engine. The former executes the running program and accesses its runtime state to get the required information (consist of variables and their value). This information contains a subset of relevant variables and only these are written to the trace file. A variable is considered as relevant depending on the type of invariants targeted and whether they are accessible at an instrumentation point. This forms an input to the second part of the system - the invariant inference engine - which uses machine learning techniques and produces a set of detected invariants for the program.

The main challenge with these techniques is deducing the relevance of the invariants as it depends on the programmer's experience and knowledge of the underlying system. However, these can be improved with techniques like exploiting unused polymorphism and suppressing invariants that are logically implied by other invariants. DAIKON does not require the programmer to specify invariants for the application, however, it is not designed for distributed or resource-constrained systems like WSN.

\paragraph{DIDUCE} One of the issues with DAIKON is that it only detects possible invariants, DIDUCE \cite{diduce} (Dynamic Invariant Detection $\cup$ Checking Engine (DIDUCE)) uses a similar metholody to detect the invariants, but it also evaluates them as they are detected. Machine learning is employed to dynamically generate hypotheses of invariants for a system at run-time. The invariants begin extremely strict, and are relaxed over time to allow for new correct behaviour. The machine learning aspect means that developers do not have to specify invariants themselves (as compared to H-SEND), which proves beneficial as accurately pinpointing the values necessary for fault-free operation is non-trivial \cite{?}. DIDUCE checks against the invariants continually during a program's operation and reports all violations detected at the end of the run, whereas DAIKON merely presents the user with invariants found. For all its apparent usefulness, unfortunately DIDUCE was designed for large, complex systems rather than lightweight distributed systems with constrained resources such as sensor networks.

\paragraph{DICAS} Moving away from automatically detecting invariants, in wireless sensor networks security is very important. As well as more traditional network attacks that the network is vulnerable, the network can be attacked physically \cite{Perrig04securityin} or the context of where messages originate from can be exploited to reveal the physical location of a source of information \cite{Ozturk:2004:SPE:1029102.1029117,Kamat.1437121,6296046}. DICAS (Detection, Diagnosis and Isolation of Control Attacks in Sensor Networks) \cite{dicaspaper} is a lightweight distributed protocol for Wireless Sensor Networks that mitigates the effects of a wide-range of control traffic attacks. DICAS does this by utilizing a unique property of a wireless sensor network, this is that each node is able to monitor a part of their neighbour's network traffic. Using DICAS the authors were able to create LSR a lightweight secure routing algorithm for Wireless Sensor Networks.

In DICAS, nodes maintain a data structure of their first hop neighbours for local monitoring to detect malicious nodes and in local response to isolate these nodes. When a node is deployed it finds and authenticates all of its 1 hop neighbours using a pairwise shared key. These neighbours communicate their neighbours, 2 hop neighbours to the original node, and their own commitment key, which is generated using a random seed, to the newly deployed node. Each node will then have knowledge of all of its 1 hop neighbours as well as their commitment keys and all of its 2 hop neighbours.

Using this knowledge the DICAS protocol can enact a collaborative detection strategy where every node monitors the traffic going in and out of it's neighbours. Each node that is within transmission range of both the sending(X) and receiving(A) nodes of a packet are considered guard nodes of A over the link from X to A. These nodes maintain a watch buffer of packets sent from X to A, the duration and information stored are determined by the attack type under consideration by the system. Each guard node maintains a malicious counter for each link it is monitoring, if A drops, delays, changes or fabricates a packet from X then the guard node will increase its malicious counter. If the malicious counter exceeds a predefined threshold the node is considered to be malicious and is removed from the neighbours list. The guard node propagates this knowledge to all the other nodes in its neighbour list. When another node receives enough authenticated alerts about the malicious node it is excluded from its neighbours list. Once all 1 hop nodes have excluded the malicious node then it is effectively isolated from the system and all packets from or to that node are ignored.

The authors tested the DICAS algorithm against these 5 sets of attacks:
\begin{enumerate}
\item Route Traffic Manipulation
\item ID Spoofing and Sybil Attacks
\item Wormhole Attacks
\item Sinkhole
\item Rushing Attack
\end{enumerate}

Additionally, the paper describes a cost analysis on the DICAS algorithm, these are the results: 
\begin{itemize}
\item Memory Overhead

The memory overhead is the most pressing of the overheads created by the DICAS algorithm. The algorithm must store several data structures on each node: a neighbours list, a watch buffer, a commitment key list and the alert list. These structures are variable on sizes dependent on the number of nodes in the network, the network layout and the MAC layer delay for acquiring a channel. Any implementation must consider the memory cost of the algorithm seriously.

\item Computation Overhead

With regards to computation they found that each packet received or sent required: one lookup for the current source and destination in the neighbour list, for an incoming packet - adding an entry to the watch buffer or for an outgoing packet - deleting an entry from the watch buffer. Since the size of the watch buffer and the neighbour list structure are relatively small, the computation time required for these operations is negligible.

\item Bandwidth Overhead

When considering bandwidth the overhead was primarily gained in 2 conditions: after node deployment when a node is populating its neighbour list and during a wormhole attack detection where a node is informing its neighbours of the malicious node. However, these cases make up a negligible fraction of the total network traffic over the lifetime of a wireless sensor network.
\end{itemize}

\subsubsection{Practical Sensor Network Deployments}


\subsubsection{Tools and Platforms}


\section{Literary Review}

\subsection{Quality of Service}
The area of Quality of Service (QoS) within Wireless Sensor Networks (WSN) is largely unexplored, due to the large differences between WSNs and traditional wireless networks. Traditional networks determine QoS based on high bandwidth allowance, as a result of high multimedia demands of applications. WSNs typically do not need to transfer this amount of data, and have a much lower bandwidth because of this. WSNs also have a wide range of different applications, and as a result, it is not clear how to develop transferable approaches to QoS  \cite{Akyildiz2002393}. 

QoS can be reduced to 'a set of service requirements to be met when transporting a packet stream from the source to its destination' \cite{Crawley98aframework}. With traditional networks, redundancy is often introduced to allow for high load/traffic, however redundancy in WSNs can often mean wasted energy usage which is often the main QoS measure in many protocols \cite{AkkayaYounis2003}.

Akyildiz et. al. \cite{Akyildiz2002393} suggested that QoS could be measured in two ways Application and Network. The Application defines measures such as coverage, number of active sensors and exposure, while the Network is concerned with delivering the QoS constrained data, while maintaining network efficiency (minimising resources).

Akyildiz et. al. further went on to describe the challenges specific to WSN; 
\begin{itemize}  
			\item Resource Constraints - Battery life, memory, bandwidth etc
			\item Unbalanced traffic - Traffic flows from large set of sensors into a small set of sink nodes
			\item Data redundancy - The re-transmission of data could result in wasted energy usage
			\item Network Dynamics - Failing nodes/wireless links, energy conservation, mobility etc
			\item Scaleability 
			\item Multiple sink nodes - Each node could have a different set of requirements
			\item Packet Critically - Some data may need to flow through the network quicker than other pieces
			\item Multiple traffic types - Different pieces of data flowing through the network at the same time
\end{itemize}

QoS is a difficult term to define, mainly due to its various meanings and perspectives, because of this, measurements of quality must be generated based on the application involved, and the specific requirements of that application.
